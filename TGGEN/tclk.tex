\subsection{Tested Clocks}
It is helpful to identify those clock variables of a clocked process term
or network which are relevant in determining the initial behaviour of a 
system. Such clocks are said to be \emph{tested}. The following example
illustrates the difference between tested and untested clocks.
\begin{exampleb}
Consider the term $[ReadSensor:5]^{H1} \sq \kk!flow.\xx$.
It is clear that only the value of the clock variable $H1$ influences
the initial behaviour: if the value of $H1$ is 5 then a $ReadSensor$
transition is possible, otherwise it is not.  The value of $H2$ has no
effect on behaviour until after a $ReadSensor$ transition has
occurred.
\qed
\end{exampleb}
In order to account properly for those clock variables
which are associated with a data guard, it is necessary to consider
the data environment in which the guard is evaluated: if the guard
evaluates to $\True$, its clock is tested, otherwise it is not tested.
This leads to the following definition:
\begin{definition}[Tested Clock Variables: Process Term]
\label{def:tgtclkproc}
The tested clock variables of a process term, with
respect to a given data environment, are identified by
the function $\tclk : \CProc{} \cross \DataEnv{} \fun 2^\clocks$,
defined as the least set satisfying:
\begin{eqnarray*}
\tclk(\kk?\ii.\xx,\D) & = & \emptyset \\
\tclk(\kk!\ii.\xx,\D) & = & \tclk([\op:t_1,t_2]^\clock,\D) = \{\clock\} \\
\tclk(\g\guard\PP_1,\D) & = & \ite{\D\models\g}{\{\clock\}}{\emptyset} \\
\tclk(\PP_1 \sq \PP_2,\D) & = & \tclk(\PP_1,\D) \\
\tclk(\PP_1 @ \PP_2,\D) & = & \tclk(\PP_1,\D) \cup \tclk(\PP_2,\D), \quad @\; \in
\{\choice,\interrupt,\parallel\} \\
\tclk(\rec X.\PP,\D) & = & \tclk(\PP[\rec X.\PP/X],\D) 
\end{eqnarray*}
\qed
\end{definition}
\qed
\end{definition}

The notion of \emph{tested} clock is useful also in the context of 
networks, where the clock associated with some network channel $\kk$ is
said to be \emph{tested} if its current
value is relevant to the immediate evolution of the system containing
it. We can see that a channel clock is tested \emph{unless} its channel is 
currently free and has no messages pending transmission.
\begin{definition}[Tested Clock Variables: Network]
The tested clocks of a network are given by the function 
$\tclk : \NNetwork{} \fun 2^\clocks$, where
$ \tclk(\NN) \defs \clk(\NN) \setminus 
        \{\clock_\kk | (\free,\emq)_\kk \in \NN\} $
\qed
\end{definition}
  
