% $Id: abstract.tex,v 1.3 1997/10/19 23:22:56 davek Exp davek $
\begin{abstract}
Embedded systems are \emph{real-time, communicating} systems, and the
effective modelling and analysis of these aspects of their behaviour
is regarded as essential for acquiring confidence in their correct
operation. In practice, it is important to minimise the burden of
model construction and to automate the analysis, if possible. Among
the most promising techniques for real-time systems are
\emph{reachability analysis} and \emph{model-checking} of networks of
\emph{timed automata}.  We identify two obstacles to the application
of these techniques to a large class of distributed embedded systems:
firstly, the language of timed automata is too low-level for
straightforward model construction, and secondly, the synchronous,
handshake communication mechanism of the timed automata model does not
fit well with the asynchronous, broadcast mechanism employed in many
distributed embedded systems. As a result, the task of model
construction can be unduly onerous.

This dissertation proposes an expressive language for the construction
of models of real-time, broadcasting control systems, and demonstrates
how efficient analysis techniques can be applied to them.

The dissertation is concerned in particular with the Controller Area
Network (CAN) protocol which is emerging as a \emph{de facto} standard
in the automotive industry.  An abstract formal model of CAN is
developed. This model is adopted as the communication primitive in a
new language, \bcandle, which includes value passing, broadcast
communication, message priorities and explicit time.  A high-level
language, \candle, is introduced and its semantics defined by
translation to \bcandle. We show how realistic CAN systems can be
described in \candle\ and how a timed transition model of a system can
be extracted for analysis. Finally, it is shown how efficient methods
of analysis, such as `on-the-fly' and symbolic techniques, can be
applied to these models. The dissertation contributes to the practical
application of formal methods within the domain of broadcasting,
embedded control systems.
\end{abstract}




