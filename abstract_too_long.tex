% $Id: abstract.tex,v 1.3 1997/10/19 23:22:56 davek Exp davek $
\begin{abstract}
Embedded systems are \emph{real-time, communicating systems}, and the
effective modelling and analysis of these aspects of their behaviour
is regarded as essential to increasing confidence in their correct
operation. From a practical point of view, it is important to minimise
the burden of model construction and to automate the analysis, if
possible. Among the most promising techniques for automated analysis
of real-time systems are \emph{reachability analysis} and
\emph{model-checking} of networks of \emph{timed automata}.  We
identify two obstacles to the application of these techniques to
embedded systems: 1) the language of timed automata is low-level, and
2) the synchronous, handshake communication mechanism of the timed
automata model differs significantly from the asynchronous, broadcast
mechanism employed in the implementation of many embedded systems.

This dissertation demonstrates that it is possible to develop models
of real-time, broadcasting control systems, using an expressive
modelling language with a broadcast communication primitive, while
also allowing the use of the latest, efficient analysis techniques,
which have been developed in the context of timed automata. The
dissertation is concerned in particular with the Controller Area
Network (CAN) protocol which is emerging as a
\emph{de facto} standard in the automotive industry.  An abstract
formal model of CAN is presented. This model is adopted as the
communication primitive in a new language, \bcandle, which includes
value passing, broadcast communication, message priorities and
explicit time.  A high-level language, \candle, is introduced and its
semantics defined by translation to \bcandle. We show how realistic
CAN systems can be developed in \candle\ and how a timed transition
model of a system can be extracted for analysis. Finally, it is shown
how many of the most recent advances in the analysis of timed systems,
in particular `on-the-fly' and symbolic techniques, can be applied to
these models.  Thus, the dissertation extends the scope of the
practical application of promising formal methods to the domain of
broadcasting, embedded control systems.
\end{abstract}


