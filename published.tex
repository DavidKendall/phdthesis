\chapter*{Published Work}
Preliminary versions of some of the work in this thesis have been presented at
a number of conferences and workshops. Steven Bradley, William
Henderson and Adrian Robson are co-authors of many of the following
papers.  The work presented in the thesis is entirely my own, except where
explicitly acknowledged. The papers, in chronological order, are
\begin{itemize}
\item
\newblock A formal basis for tool-supported simulation and verification of
  real-time {CAN} systems.
\newblock In {\em Proceedings of 4th International {CAN} Conference
  {(iCC'97)}}, pages 719--727, Berlin, October 1997. 

\item
\newblock \bcandle: Formal modelling and analysis of {CAN} control systems.
\newblock In {\em Proceedings of 4th IEEE Real Time Technology and Applications
  Symposium (RTAS'98)}, pages 171--177. IEEE Computer Society Press, June 1998.

\item
\newblock \candle: A high level language and development environment for high
  integrity {CAN} control systems.
\newblock In {\em Proceedings of 4th IEE Workshop on Discrete Event Systems}, 
  pages 58--63, August 1998.

\item 
\newblock Using sharing trees in the automated analysis of real-time systems 
  with data.
\newblock In {\em Proceedings of IEE Colloquium: Applicable Modelling, 
  Verification and Analysis Techniques for Real-Time Systems}, 
  Ref. No.1999/006, pages 6/1-4. IEE, London, UK, January 1999.

\item
\newblock \candle: A tool for efficient analysis of CAN control systems.
\newblock In {\em Proceedings of the 1st Workshop on Real-Time Tools 
  (RT-TOOLS'2001), Aalborg, Denmark}, Technical Report 2001-014, 
  University of Uppsala, August 2001.
\end{itemize}

My ideas concerning the translation from a process language to timed automata
were developed first for the AORTA language. That work appears in
\begin{itemize}
\item 
\newblock Validation, verification and implementation of timed protocols using
  {AORTA}.
\newblock In P.~Dembinski, editor, {\em Proceedings of the Fifteenth
  International Symposium on Protocol Specification, Testing and Verification},
  pages 205--220. Chapman and Hall, June 1995.
\end{itemize}
