\documentclass{article}
\usepackage{a4,amsmath,becs}
\title{Commentary on thesis corrections}
\author{David Kendall}
\begin{document}
\maketitle
\section{Corrections proposed by ZL}
\begin{trivlist}
\item[\bf GQ 1--3] No action required.
\item[\bf GQ 4] Added a couple of sentences to end of para.1 on p.147.
\item[\bf GQ 5] The first question was addressed at the beginning of
\Sec6.2, pp.145--146. I've added the word 'automatically' in a couple
of places in para.1 of this section to emphasise this point. I've
addressed the second point by adding a sentence to para.1 of
\Sec6.2. I take the third point about code generation to be a question
about how a model can be constructed from an implementation which
includes such features as exceptions etc., while still accurately
capturing timing characteristics. \Sec6.4 is intended to deal with
these issues for a kind of 'generic' implementation platform. It
provides enough detail to suggest how such features can be modelled
automatically, I think. More detail would depend upon developing much
more specific assumptions about the implementation platform, which I
have said is outside the scope of the thesis.
\item[\bf GQ 6] The approach has been tried only on the examples in the
thesis, plus one or two others of comparable size and complexity. I
had tried to acknowledge this in the final sentence of \Sec6.7.1 on p.186. I
have now extended this sentence to make clearer the limitations and
need for further work.
\end{trivlist}

\begin{trivlist}
\item[\bf DQ 1] Line -15 on p.4: paragraph rephrased as suggested.
\item[\bf DQ 2] Line -5 on p.7: fixed.
\item[\bf DQ 3] Last para \Sec2.1. I didn't understand the problem
here. I \emph{think} what is written already conforms with ZL's suggestions.
\item[\bf DQ 4] Line 17 on p.12: I've checked the formula and think it's
OK. The second conjunction is just the standard definition of a dense
domain, see [Koy91], for example. The version presented here is as
given in [Nic92] and is a little more pedantic in requiring that a
dense domain is not a trivial one point domain. $\Time$ clearly
satisfies this requirement also. I've added an explanation of the sets
to the beginning of \Sec2.2.
\item[\bf DQ 5] p.12: added example of cyclic circuits and reference
to [AMP98].
\item[\bf DQ 6] Line -20 on p.20: added paragraph immediately after
syntax of clock constraints.
\item[\bf DQ 7] Para.1 p.26: changed to 'encoded as state formulas'.
Line 14: fixed typo.
\item[\bf DQ 8] Line -2 on p.30: inserted forward reference to example
property. Line 18: fixed typo.
\item[\bf DQ 9] I've assumed that this request has been deleted? It's
rather a large topic, and somewhat outside the scope of the thesis.
\item[\bf DQ 10] Line 14 on p.42: Checked. Greatest is correct. The
smallest polyhedron is just $\vexcond$.
\item[\bf DQ 11] Page 51, 2nd bullet: expanded this paragraph to include
more explanation.
\item[\bf DQ 12] Para. 3 line 2 on p.59: added a sentence of further 
explanation and proposed temporal logic for specification.
\item[\bf DQ 13] Line 3 on p.60: added a brief explanation about the
constraints on the schedulers that can be handled.
\item[\bf DQ 14] Line 13 on p.60: added an indication (with
references) of how a CAN-style channel can be implemented using shared
memory techniques for communication between processes which share a
processor.
\item[\bf DQ 15] p.60 bullet 6: added clarification of what is meant
by no interference between channels.
\item[\bf DQ 16]Line 15 on p.62: added a short justification for the use of
general relations.
\item[\bf DQ 17--20] Adopting these suggestions would lead to some
improvement in the presentation of channels and networks but would
also have implications for the presentation throughout the rest of the
thesis. Therefore, I have deferred the adoption of this approach to
future work.
\item[\bf DQ 21] p.76, termination: I've considered the inclusion of a
successful termination operator in the syntax, and the algebraic laws
which it might satisfy, long and hard. This still seems like a very
difficult issue to treat properly and not one which I would like to
tackle at this stage in the thesis. I've done some more reading since
the viva and I've found a quite recent paper by Baeten and Reniers which
I've now cited in the thesis. They have tried to deal with termination
properly in a setting which is not too dissimilar from mine. They have
found it necessary to introduce \emph{two} forms of unsuccessful termination
and \emph{two} forms of successful termination and to adopt a non-standard
approach to the definition of strong bisimilarity in order to cope with
what they call the \emph{intricate interplay between termination, time
determinism, and sequential composition}. If I adopt the semantics
suggested by ZL for $\tick$ then it behaves more like unsuccessful than
successful termination, e.g. $\tick \sq P \bisim \tick$ and it seems to be
likely that bisimulation is no longer a congruence. For these reasons, I
have just added a section on termination which mentions the problems and 
refers to the literature (p.83 in the new version of the thesis) but I've
otherwise left things unchanged.

Precedence of operators: The unusual precedence
ordering in which $\parallel$ is weaker than $\sq$ and $\choice$ is
adopted because one almost always wants to write expressions with
parallelism at the top-level in \bcandle. I've added $\rec$ to the
precedence list. It's position, too, is chosen to reduce the need for
parentheses in the most common forms of \bcandle\ expression. I've left
the introduction of the abbreviations where they are, since this is a
matter of syntax rather than semantics, but I've also added a couple
of sentences to the description of 'Compute' in the informal semantics
to discuss the null and idle processes.
\item[\bf DQ 22] Definition 3.21 on p. 78: I've removed the equality (=) and
replaced it by a rewrite relation ($\goesrw$) to emphasise that
this is just syntactic rewriting. Example 3.4 has been modified accordingly.
\item[\bf DQ 23] p.83: in fact the syntactic and semantic equalities
are the same, but I've changed to $P \nsyneq \tick$ as suggested. This change
needs to be propagated throughout the thesis. In fact, only the proof in 
Appendix B is affected and the necessary changes have been made there also.
\item[\bf DQ 24] p.84, semantics of interruption: I've tried to
address this issue by including some further explanation and an example
in the informal semantics on p.79. I've done this by adding further explanation
about the choice operator and then relating the behaviour of interrupt to 
choice. See \textbf{DQ 21} for issues relating to $\tick$.
\item[\bf DQ 25] Definition 3.29 on p.85: semantic equality is intended. I've 
added a definition of data environment equality to the end of \Sec3.3 and
definitions of channel and network equality immediately after Definitions 3.8
and 3.9, respectively.
\item[\bf DQ 26] top of p.86: added a brief statement to the end of para.1 of
\Sec3.6.4 to point out the lack of a complete axiomatisation and expansion
theorem.
\item[\bf DQ 27] p.127: this is the same algorithm as that considered in 
DQ 11. I think the extra explanation there should be enough.
\item[\bf DQ 28] Proposition 5.1 on p.130: expanded the final para of
\Sec5.2 to discuss the scope and implications of the proposition.
\end{trivlist}
\section{Corrections proposed by AY}
I don't have a list of Alex's suggestions. I did make some notes during
the viva. Many of his suggestions have already been considered in the
previous section, or are considered in the next section. The few remaining
are discussed below.
\begin{trivlist}
\item[\bf AY 1] p.110: \textbf{S} - Instead of using $\WTick$ include
$\tick$ in $\W$. \textbf{R} - At the viva, I agreed that this is a
good idea. In fact, I'd already done this myself for the RT-TOOLS
paper. However, I now realise that this change would involve changing
all of the net definitions in \Sec4.4.3 (I didn't have this problem
for the paper because this material was omitted). Since the existing
presentation is consistent, I'd prefer to leave it as it is and adopt this
idea in future publications.
\item[\bf AY 2] p.116, Parallel Composition: \textbf{S} - Explain why the
restriction of parallelism to top-level is useful to the
implementation. \textbf{R} - I've added a paragraph to the end of this section.
\item[\bf AY 3] p.123: \textbf{S} - Say something about the complexity of the 
translation. \textbf{R} - I've added a sentence on this topic to the first para
on this page.
\end{trivlist}
\section{Corrections proposed by DK}
\begin{trivlist}
\item[\bf DK 1] Added RT-TOOLS paper to list of published work.
\item[\bf DK 2] p.68: Delete 'message' in row 5 col 3 of Fig. 3.2.
\item[\bf DK 3] p.70: 'A network N over K and I ...' to 'A network N, over 
  K and I, ...'.
\item[\bf DK 4] p.92: add 'latent' and 'dense' to last sentence of para.2.
\item[\bf DK 5] p.102, line 4: formatting.
\item[\bf DK 6] p.102 ff: as currently defined the TA $\Gr(\csys)$ is
not a TA since it has an infinite number of locations! This is just a
temporary blip, since a finite number of locations is obtained by the
restriction to locations which are structurally reachable from the
initial location. The revision is intended to remove this anomaly. The concept
of structurally reachable location is now introduced earlier and $\Gr(\csys)$
is defined as $\sreach(\GrPlus(\csys))$, where $\GrPlus(\csys)$ is just the
old definition of $\Gr(\csys)$. There are a few places later in the thesis,
particularly in Appendix~B, where this change needs to be accommodated.
\item[\bf DK 7] p.110, line 5: formatting.
\item[\bf DK 8] p.110, line -12: '$\tr_\ww$' to '$\preset{\tr}$'.
\item[\bf DK 9] p.120, Definition 4.27, line 2: fixed bracket mismatch.
\item[\bf DK 10] p.122, lines -12 and -11: swapped $\dlB$ and $\dub$.
\item[\bf DK 11] p.127 ff: changed font from $\RR$ to \textbf{R} for reference 
to rules \textbf{R.1} and \textbf{R.2}.
\item[\bf DK 12] p.132, Example 5.1, line 2: formatting.
\item[\bf DK 13] p.135, line -7: added reference to [Tri98].
\item[\bf DK 14] p.160, line 4: line break
\item[\bf DK 15] p.160, line 13: added 'assuming that the index type of
  \verb'a' is \verb-0..3-'.
\item[\bf DK 16] p.180, Fig.6.6(b): added \verb'type pump_status' to 
  \verb'module Pump'.
\item[\bf DK 17] p.183, Fig. 6.9: fixed bounds in network section which don't
comply with the scenario described in the text.
\item[\bf DK 18] p.184, line 3 ff: changed to results as given by modified
  network section (see DK 17).
\item[\bf DK 19] p.195, Proposition B.2: 'If R is a strong bisimulation ...'
  to 'If R is a strong $\approx$-bisimulation ...'.
\item[\bf DK 20] p.198, Lemma B.4: qualified the scope of the lemma to
  extend only to \emph{compatible} data environments since the original
  scope was too wide (Compatibility is defined at the end of \Sec3.3.1
  along with equality).
\item[\bf DK 21] p.200, line -5ff: tightened presentation of proof a little.
\item[\bf DK 22] p.203, line 11: changed $\NN$ to $\NN'$
\item[\bf DK 23] p.206, line 25: changed $\NN$ to $\NN'$
\end{trivlist}
\end{document}











