% $Id: appendixa.tex,v 1.1 1997/10/19 19:24:39 davek Exp davek $
\chapter{Proofs}\label{app:proofs}
\section{Correctness of the translation}
This section is concerned with demonstrating the correctness of the 
translation of \bcandle\ system models to timed automata. Its
purpose is to prove the validity of Proposition~\ref{prop:tgcorrect},
which we state again here.
\begin{proposition}\label{prop:pftgcorrect}
Let $\csys \in \CSys{}$ be a clocked \bcandle\ system and
$\sys \defs \unclk(\csys)$ the corresponding unclocked system.
Let $\Gr(\csys)$ be the TA given by 
Definition~\ref{def:tgconstruct}. Then, the transition systems
of $\Gr(\csys)$ and $\sys$ are strongly equivalent.
\[
\TSys{\Gr(\csys)} \bisim \TSys{\sys}
\]
\end{proposition}

The proof of the proposition depends upon demonstrating the existence of
a strong bisimulation relation between $\TSys{\Gr(\csys)}$ and $\TSys{\sys}$. 
A number of auxiliary definitions and lemmas are required.

Firstly, we define the set of states which can occur in the
transition system of a clocked \bcandle\ system, where it is required
that in any such state $(\csys,\clkvl)$, the clock valuation $\clkvl$ satisfies
the location invariant $\tginv(\csys)$.
\begin{definition}
Let $\clocks$ be a set of clock variables and
$\csys \in \CSys{}$ a clocked \bcandle\ system whose clocks are taken from
the set $\clocks$, i.e. $\clk(\csys) \subseteq \clocks$. It is assumed that
$\clocks$ also contains the distinguished urgent clock $\uclock$,
i.e. $\uclock \in \clocks \setminus \clk(\csys)$. We denote
by $\States_{\CSys{},\clocks}$ the states of the transition system of the
TA for $\csys$, i.e. \\
\hspace*{2cm} $\States_{\CSys{},\clocks} \defs 
  \{(\csys,\clkvl) | \csys \in \CSys{} \land \clkvl \in \clkvls \land
                     \clkvl \models \tginv(\csys)\}$
\qed
\end{definition}

Now, a clocked \bcandle\ system $\csys$ is related to an equivalent 
\bcandle\ system $\sys$, by the notion of \emph{aging}:
\begin{definition}[Aging]\label{def:pfage}
Let $\clocks$ be a set of clock variables.
Let $\csys \in \CSys{}$ be a clocked \bcandle\ system
where $\clk(\csys) \subseteq \clocks$. Then, 
$\age : \States_{\CSys{},\clocks} \fun \Sys{}$ 
is a function giving an aged \bcandle\ system, where
\[\age((\PP,\NN,\D),\clkvl) \defs
(\age(\PP,\clkvl),\age(\NN,\clkvl),\D).\]

The function $\age : \CProc{} \cross \clkvls \fun \Proc{}$ is defined by
\begin{eqnarray*}
\age(\kk!\ii.\xx,\clkvl) & \defs & \kk!\ii.\xx \\
\age(\kk?\ii.\xx,\clkvl) & \defs & \kk?\ii.\xx \\
\age([\op:\ti_1,\ti_2]^\clock,\clkvl) & \defs & [\op:\ti_1\nmin\clkvl(\clock),\ti_2\nmin\clkvl(\clock)] \\
\age(\g\guard\PP,\clkvl) & \defs & \g\guard\unclk(\PP) \\
\age(\PP\sq\QQ,\clkvl) & \defs & \age(\PP,\clkvl) \sq \unclk(\QQ) \\
\age(\PP\genop\QQ,\clkvl) & \defs & \age(\PP,\clkvl) \genop \age(\QQ,\clkvl),\qquad \genop\; \in \{\choice,\interrupt,\parallel\} \\
\age(\rec X.\PP,\clkvl) & \defs & \age(\PP[\rec X.\PP/X],\clkvl)  
\end{eqnarray*}
As usual, we rely on the fact that $\PP$ is guarded to ensure that
$\age(\PP,\clkvl)$ is well-defined.

The function $\age : \NNetwork{\K} \cross \clkvls \fun \Network{\K}$
is defined by 
\[\age(\NN,\clkvl) \defs 
\{\kk \mapsto \age(\NN_\kk,\clkvl) | \kk \in \K\}\] 
where
\begin{eqnarray*}
\age((\free,\mq)^\clock,\clkvl) & \defs & (\free,\mq) \\
\age((\preact{\ti_1,\ti_2}{\mm},\mq)^\clock,\clkvl) & \defs & (\preact{\ti_1\nmin\clkvl(\clock),\ti_2\nmin\clkvl(\clock)}{\mm},\mq) \\
\age((\offers{\mm},\mq)^\clock,\clkvl) & \defs & (\offers{\mm},\mq) \\
\age((\postact{\ti_1,\ti_2}{\mm},\mq)^\clock,\clkvl) & \defs & (\postact{\ti_1\nmin\clkvl(\clock),\ti_2\nmin\clkvl(\clock)}{\mm},\mq) 
\end{eqnarray*}
\qed
\end{definition}

The main proof makes use of the standard technique of demonstrating
a strong bisimulation \emph{up to} $\bisim$. It is necessary to adapt 
the usual notion of strong bisimulation up to $\bisim$ to 
$\approx$-bisimulation.
\begin{definition}[Strong bisimulation up to $\bisim$]
Let $\SS = (\States, \Init, \Labels, \TRel)$ be a LTS. Let $\approx$ be an
equivalence relation on $\States$.  A binary relation $R \subseteq
\States \cross \States$ is a strong $\approx$-\emph{bisimulation} up to
$\bisim$ if $\state_1 R \state_2$ implies
\begin{enumerate}
\item $\state_1 \approx \state_2$ 
\item for all $\any \in \Labels$, if $\state_1 \goes{\any} \state_1'$, then
  $\state_2 \goes{\any} \state_2'$ for some $\state_2'$ such that
  $\state_1' \bisim R \bisim \state_2'$
\item for all $\any \in \Labels$, if $\state_2 \goes{\any} \state_2'$, then 
  $\state_1 \goes{\any} \state_1'$ for some $\state_1'$ such that
  $\state_1' \bisim R \bisim \state_2'$
\qed 
\end{enumerate} 
\end{definition}

\begin{proposition}
If $R$ is a strong $\approx$-bisimulation up to $\bisim$, then $\bisim
R \bisim$ is a strong $\approx$-bisimulation.
\end{proposition}
\begin{proof}
As the proof of Lemma~4.5 in Milner~\cite{mil:89}.
\end{proof}

In order to apply the notion of strong bisimulation in the context of
the main proof, it is necessary to extend 
$\ndequiv$-bisimulation to clocked states and
to pairs of clocked and unclocked states. This requires
us to revise our definition of context equivalence.
\begin{definition}[Context Equivalence]\label{def:pfcontextequiv}
Let $\state_1, \state_2 \in \Sys{} \cup \States_{\CSys{},\clocks}$ be either 
clocked or unclocked \bcandle\ system states. We denote by 
$\state_1 \ndequiv \state_2$
that $\state_1$ is \emph{context equivalent} to $\state_2$, and
define $\ndequiv \subseteq \Sys{} \cup \States_{\CSys{},\clocks} \cross 
\Sys{} \cup \States_{\CSys{},\clocks}$ by requiring that
$\state_1 \ndequiv \state_2$
iff one of the following conditions is satisfied:
\begin{enumerate}
\item $\state_1 = (\P_1,\N_1,\D_1) \in \Sys{}$,
      $\state_2 = (\P_2,\N_2,\D_2) \in \Sys{}$, $\N_1 = \N_2$ and $\D_1 = \D_2$
\item $\state_1 = (\P,\N,\D) \in \Sys{}$, $\state_2 = ((\PP,\NN,\DD),\clkvl)
  \in \States_{\CSys{},\clocks}$, $\N = \age(\NN,\clkvl)$ and $\D = \DD$
\item as (2) with the roles of $\state_1$ and $\state_2$ reversed
\item $\state_1 = ((\PP_1,\NN_1,\DD_1),\clkvl_1) \in 
  \States_{\CSys{},\clocks}$, $\state_2 = ((\PP_2,\NN_2,\DD_2),\clkvl_2) \in 
  \States_{\CSys{},\clocks}$,
  $\age(\NN_1,\clkvl_1) = \age(\NN_2,\clkvl_2)$ and $\DD_1 = \DD_2$.
\qed
\end{enumerate} 
\end{definition}

\begin{proposition}
$\ndequiv$ is an equivalence relation.
\end{proposition}
\begin{proof}
Immediate from Definition~\ref{def:pfcontextequiv}.
\end{proof}

Strong equivalence of both clocked and unclocked \bcandle\ states is defined
simply as $\ndequiv$-bisimilarity, and related definitions are obtained in
the obvious way.
\begin{remark}
Notice that Propositions~\ref{prop:bccongruence} and~\ref{prop:bcsound}
are valid when extended to clocked systems, i.e. $\bisim$ is 
a congruence for the operators of $\CProc{}$ and the equational laws
are sound for $\CSys{}$. 
\end{remark}

Now, we turn our attention to addressing a technical point concerning
the use of $\tick$. We wish to obtain a compositional proof and to
avoid the need to reason about the persistence of systems. In order to
achieve this, it is convenient to treat $\tick$ as a (distinguished)
process term and to introduce locations $(\tick,\NN,\DD)$
corresponding to systems $(\tick,\N,\D)$. We silently extend $\Sys{}$
and $\CSys{}$ to contain these additional systems. The definition of
$\age$ is extended by $\age(\tick,\clkvl) \defs \tick$, and the
invariant function $\tginv$ by $\tginv(\tick,\NN,\DD) \defs
\uclock \leq 0 \land \tginv(\NN)$. Now, as currently
defined, a system $(\tick,\N,\D)$ has no transitions. But if a clock 
valuation $\clkvl$ satisfies $\tginv(\tick,\NN,\DD)$, then
$((\tick,\NN,\DD),\clkvl)
\goes{0} ((\tick,\NN,\DD),\clkvl)$. To resolve this
discrepancy, we assume that the semantics of \bcandle\ is
extended with a rule
\[
\etick
\]
Now it is clear that $((\tick,\NN,\DD),\clkvl[\uclock:=0])$ is
strongly bisimilar to $\age((\tick,\NN,\DD),$ $\clkvl[\uclock:=0])$, each
state having only a 0-transition to itself.

The following proposition asserts that $\tick$ really is a distinguished
process term.
\begin{proposition}\label{prop:pftickdistinguished}
For any \bcandle\ system $(\P,\N,\D) \in \Sys{}$, 
\[\text{if } (\P,\N,\D) \bisim (\tick,\N,\D), \text{then } \P \syneq \tick\;.\]
\end{proposition}
\begin{proof}
A standard induction. Intuitively, we can see from the
semantics that $\tick$ is the only process which does not allow either 
the immediate execution of some discrete action or the passage of time
by some strictly positive amount.  
\end{proof}

We now introduce several lemmas which will be useful in the main proof.
Each lemma is introduced by a few words of informal explanation.

The first two lemmas simply assert that network behaviour is both
\emph{independently determined} and \emph{non-intrusive} in both clocked and
unclocked systems. Network behaviour is independently determined in
the sense that each new network state is uniquely determined by a
current network state and a network action, irrespective of the system
context. It is non-intrusive in that the process and data components
of a system state are unchanged by network transitions.
\begin{lemma}\label{lem:pfnetdeterm}
Let $(\P_1,\N,\D_1), (\P_1',\N',\D_1'), (\P_2,\N,\D_2)$ and 
$(\P_2',\N'',\D_2')$ be \\ \bcandle\ system 
states in $\Sys{}$. Let $\any \in \NetLabels \cup \Time$ be a 
network transition label. Then, if
\[(\P_1,\N,\D_1) \goes{\any} (\P_1',\N',\D_1') \text{ and } (\P_2,\N,\D_2) \goes{\any} (\P_2',\N'',\D_2')\]
we have
\begin{enumerate}
\item the new network state is uniquely determined, i.e. $\N' = \N''$, and
\item the data component is unchanged, i.e. $\D_1 = \D_1'$.
\end{enumerate}
Additionally, for $\any \in \NetLabels$, we have 
\begin{enumerate}
\item[3.] the process component is unchanged, i.e. $\P_1 \syneq \P_1'$.
\end{enumerate}
\end{lemma}
\begin{proof}
It is clear from the network rules
(Definition~\ref{def:bcnetsemantics}) that a new network state is
uniquely determined by the current state and the network action. When
included in a system context, the rules for basic systems and
data-guarded systems show that the new network state is unaffected by
the process and data components, which are themselves unchanged by the
network transition in the case of a network action (only the data
component being unchanged in the case of a strictly positive delay action).
This property is preserved by all the process operators 
(Definition~\ref{def:bcsemantics}).
\end{proof}

\begin{lemma}\label{lem:pfcnetdeterm}
Let $\csys \in \CSys{}$ be a clocked \bcandle\ system and
$\Gr(\csys) = (\tglocs,\tgiloc,\tgclks,\tgedges,\tginv)$ its corresponding TA.
Let $(\PP_1,\NN,\DD_1), (\PP_1',\NN',\DD_1'), (\PP_2,\NN,\DD_2)$ and 
$(\PP_2',\NN'',\DD_2')$ be locations in 
$\tglocs$. Let $\netw \in \NetLabels$ be a network action label.
Then, if $\tgedges$ contains edges 
\[(\PP_1,\NN,\DD_1) \goes{\clkcond_1,\netw,\resets_1} (\PP_1',\NN',\DD_1') \text{ and } (\PP_2,\NN,\DD_2) \goes{\clkcond_2,\netw,\resets_2} (\PP_2',\NN'',\DD_2')\]
we have
\begin{enumerate}
\item the new network state is uniquely determined, i.e. $\NN' = \NN''$, 
\item the process and data components are unchanged, i.e. $\PP_1 = \PP_1'$
  and $\DD_1 = \DD_1'$, and
\item the clock guards and reset sets are identical, i.e. 
  $\clkcond_1 \syneq \clkcond_2$ and $\resets_1 = \resets_2$. 
\end{enumerate}
\end{lemma}
\begin{proof}
Similar to Lemma~\ref{lem:pfnetdeterm}, using Definition~\ref{def:tgconstruct}.
\end{proof}

The next three lemmas are concerned with the effect of clock resets
on the satisfaction of location invariants.

If a clock valuation satisfies a location invariant, then any clock valuation
derived from it by resetting some clocks also satisfies the location 
invariant. 
\begin{lemma}\label{lem:pfreset}
Let $\clocks$ be a set of clock variables, $\someclks \subseteq
\clocks$ and $\clkvl \in \clkvls$ a $\clocks$-valuation. Let
$\csys \in \CSys{}$ be a clocked \bcandle\ system, where $\clk(\csys)
\subseteq \clocks$. If $\clkvl \models \tginv(\csys)$, then 
$\clkvl[\someclks:=0] \models \tginv(\csys)$.
\end{lemma}
\begin{proof}
Induction on the number of steps in the expansion of $\tginv(\csys)$, 
using Definition~\ref{def:tginvariant}.
\end{proof}

If a clock valuation satisfies the invariant of a process term in
\emph{some} data environment, then any clock valuation derived from it
by resetting some clocks satisfies the invariant in \emph{any} compatible data
environment, provided the urgent clock $\uclock$ is reset.
\begin{lemma}\label{lem:pfinvpreservedinnewenv}
Let $\clocks$ be a set of clock variables and $\clkvl \in \clkvls$ a
$\clocks$-valuation. Let $\PP \in \CProc{}$ be a clocked process term,
where $\clk(\PP) \subseteq \clocks$. Let $\someclks \subseteq \clocks$
with $\uclock \in \someclks$. Then, if there is some data environment
$\DD \in \DataEnv{}$ such that $\clkvl \models
\tginv(\PP,\DD)$, it is the case that for all $\DD' \in \DataEnv{}$, such
that $\DD'$ and $\DD$ are compatible, we have $\clkvl[\someclks:=0]
\models \tginv(\PP,\DD')$.
\end{lemma}
\begin{proof}
By induction on the number of steps in the expansion of $\tginv(\PP,\DD)$, 
using Definition~\ref{def:tginvariant}.
\end{proof}

If the initial clocks of a process term, together with the urgent
clock $\uclock$, are all reset in some clock valuation, then the
resulting clock valuation satisfies the process term invariant.
\begin{lemma}\label{lem:pfprocinv}
Let $\clocks$ be a set of clock variables,
$\clkvl \in \clkvls$ a $\clocks$-valuation and $\PP \in
\CProc{}$ a clocked term, where $\clk(\PP) \subseteq \clocks$. 
Let $\someclks \subseteq \clocks$. Then, if $\iclk(\PP) \cup \{\uclock\} 
\subseteq \someclks$, it is the case that 
$\clkvl[\someclks:=0] \models \tginv(\PP,\DD)$, in any data environment $\DD$.
\end{lemma}
\begin{proof}
By induction on the number of steps in the expansion of $\tginv(\PP,\DD)$, 
using Definitions~\ref{def:tgiclk} and~\ref{def:tginvariant}.
\end{proof}

The remaining lemmas are concerned with properties of the $\age$
function.

Only the values of network clocks can affect the result of aging a 
network, and only the values of initial clocks can affect the 
result of aging a process term. 
\begin{lemma}\label{lem:pfage}
Let $\clocks$ be a set of clock variables and $\clkvl,\clkvl' \in \clkvls$
be $\clocks$-valuations.
\begin{enumerate}
\item Let $\NN \in \NNetwork{}$ be a clocked network. Then,
\[
(\age(\NN,\clkvl) = \N \land 
\forall \clock \in \clk(\NN) \such \clkvl'(\clock) = \clkvl(\clock)) 
\implies \age(\NN,\clkvl') = \N
\]
\item Let $\PP \in \CProc{}$ be a clocked process term. Then,
\[
(\age(\PP,\clkvl) = \P \land
\forall \clock \in \iclk(\PP) \such \clkvl'(\clock) = \clkvl(\clock)) 
\implies \age(\PP,\clkvl') = \P
\]
\end{enumerate}
\end{lemma}
\begin{proof}
Immediate from Definition~\ref{def:pfage} for $\NN$ and by 
induction on the number of steps in the expansion of $\age(\PP,\clkvl)$
for $\PP$.
\end{proof}

If the initial clocks of a process term are all reset in some clock
valuation, then the aging of the term by that clock valuation produces
a term which is equivalent to the corresponding unclocked term. 
\begin{lemma}\label{lem:pfprocage}
Let $\clocks$ be a set of clock variables,
$\clkvl \in \clkvls$ a $\clocks$-valuation and $\PP \in
\CProc{}$ a clocked term, where $\clk(\PP) \subseteq \clocks$. 
Let $\someclks \subseteq \clocks$. If $\iclk(\PP) \subseteq \someclks$, then
$\age(\PP,\clkvl[\someclks:=0]) \bisim \unclk(\PP)$.
\end{lemma}
\begin{proof}
By induction on the number of steps in the expansion of
$\age(\PP,\clkvl[\someclks:=0])$ using
Definitions~\ref{def:tgiclk}, \ref{def:tgunclk} and \ref{def:pfage}.
Intuitively, we can see that for all terms, except those containing
terms of the form $\rec X.\PP$, $\age$ and $\unclk$ give
results which are syntactically identical. In the case of $\rec
X.\PP$, $\unclk$ simply removes the clock variables, whereas $\age$ 
unwinds the recursion until there is no leading $\rec$, and then
removes the clock variables. In either case, it is clear that the results 
are equivalent.
\end{proof}

If a clock valuation can be increased by time $\ti$, while satisfying 
the invariant of a clocked network, then the corresponding aged network
allows the passage of time $\ti$.
\begin{lemma}\label{lem:pfnetprogress}
Let $\clocks$ be a set of clock variables. 
Let $\K$ be a finite set of channel identifiers and $\NN \in \NNetwork{\K}$
a network over $\K$, where $\clk(\NN) \subseteq \clocks$. 
Let $\clkvl \in \clkvls$ be a $\clocks$-valuation and 
$\ti \in \Time$. Then,
\[
\clkvl + \ti \models \tginv(\NN) \implies \ti \leq \tcp(\age(\NN,\clkvl))
\]
\end{lemma}
\begin{proof}
We assume that $\clkvl + \ti \models \tginv(\NN)$ and show that
$\forall \kk \in \K \such \ti \leq \tcp(\age(\NN_\kk,\clkvl))$.
The result follows directly from Definition~\ref{def:bctcp} and
\textbf{N.5}. The proof proceeds by case analysis on 
channel status. We illustrate for the case 
$\NN_\kk = (\preact{\ti_1,\ti_2}{\mm},\mq)^\clock$.
\begin{enumerate}
\item \case $\NN_\kk = (\preact{\ti_1,\ti_2}{\mm},\mq)^\clock$. \\
By Definition~\ref{def:pfage}, $\age(\NN_\kk,\clkvl) = (\preact{\ti_1\nmin\clkvl(\clock),\ti_2\nmin\clkvl(\clock)}{\mm},\mq)$. 
By Definition~\ref{def:bctcp}, \\
$\tcp(\preact{\ti_1\nmin\clkvl(\clock),\ti_2\nmin\clkvl(\clock)}{\mm},\mq) =
\ti_2\nmin\clkvl(\clock)$. But, if $\clkvl + \ti \models \tginv(\NN_\kk)$, then
$\clkvl + \ti \models \ite{\ti_2 \in \nat}{\clock \leq \ti_2}{\cctrue}$, and
\begin{eqnarray*}
\ti_2 \in \nat & \implies & \clkvl + \ti \models \clock \leq \ti_2 \\
               & \implies & \clkvl(\clock) + \ti \leq \ti_2, \quad \text{by Definition of } \models \\
               & \implies & \ti \leq \ti_2 - \clkvl(\clock) \\
               & \implies & \ti \leq \tcp(\age(\NN_\kk, \clkvl)) \\ 
\ti_2 = \infty & \implies & \ti_2 \nmin \clkvl(\clock) = \infty \\
               & \implies & \ti \leq \tcp(\age(\NN_\kk, \clkvl))               
\end{eqnarray*}
\end{enumerate} 
The other cases are similar.
\end{proof}

\subsection*{Main Proof}
It is now possible to state the proof of Proposition~\ref{prop:pftgcorrect}.

Let $\TSys{\GrPlus(\csys)} = (\States_1,\Init_1,\Labels_1,\TRel_1)$ and
$\TSys{\sys} = (\States_2,\Init_2,\Labels_2,\TRel_2)$.
We show that the relation $\birel$ is a $\ndequiv$-bisimulation up to
$\bisim$, where
\[\birel \defs \{((\csys,\clkvl),\age(\csys,\clkvl)) | (\csys,\clkvl) \in 
  \States_{\CSys{},\clocks}\}\]
and $\Init_1 \bisim\birel\bisim \Init_2$. The proof of the proposition follows
from Remark~\ref{rem:tgsreacheq} and the transitivity of $\bisim$.
  
To show that $\birel$ is a $\ndequiv$-bisimulation up to $\bisim$, we
reason as follows.  Let $\sstate \birel \state$.
\begin{enumerate}
\item It is clear from 
  Definitions~\ref{def:pfage}, \ref{def:pfcontextequiv} and the definition of
  $\birel$, that $\sstate \ndequiv \state$. 
\item 
  It is enough to show 
  that for all $\any \in \Labels$, 
  if $\state = (\P,\N,\D) \goes{\any} (\P',\N',\D')$, then 
  $\sstate = ((\PP,\NN,\DD),\clkvl) \goes{\any} ((\PP',\NN',\DD'),\clkvl')$,
  and there exist $\PP'' \bisim \PP'$ and $\P'' \bisim \P'$ such that  
  $((\PP'',\NN',\DD'),\clkvl') \birel (\P'',\N',\D')$.
  The proof is by induction on the number of steps in the
  calculation of $\age(\sstate)$. 
  We proceed by case analysis on the structure of $\sstate$.
\begin{enumerate}
\item \case $\sstate \syneq  ((\kk!\ii.\xx,\NN,\DD),\clkvl)$. \\
 So $\state \syneq (\kk!\ii.\xx,\N,\D)$, where $\age(\NN,\clkvl) = \N$
 and $\DD = \D$.

There are three sub-cases to consider:
\begin{enumerate}
\item \textbf{Snd.1}: 
$\any = \kk!\ii.\vv$, $\state' \syneq (\tick,\N',\D)$, where 
$\N_\kk = (\ss,\mq)$, $\vv = \D.\xx$, and $\N' = \assign{\N}{\kk}{(\ss,\mq\iq\ii.\vv)}$.

In this case, since $\age(\NN,\clkvl) = \N$, then $\NN_\kk =
(\hat{\ss},\hat{\mq})^\clock$, where
$\age(\hat{\ss},\hat{\mq})^\clock,\clkvl) = (\ss,\mq)$ which implies,
by Definition~\ref{def:pfage}, that $\mq = \hat{\mq}$. And, since $\DD =
\D$, then $\DD.\xx = \vv$. So by
\textbf{E\_Snd.1}, there is an edge $(\kk!\ii.\xx, \NN, \DD)
\goes{\cctrue,\kk!\ii.\vv,\{\uclock\}} (\tick,\NN',\DD)$, where 
$\NN' = \assign{\NN}{\kk}{(\hat{\ss},\mq\iq\ii.\vv)^\clock}$. Clearly, 
$\clkvl \models \cctrue$ and, since 
$\clkvl \models \tginv(\NN)$, then, by Definition~\ref{def:tginvariant},
$\clkvl[\uclock:=0] \models
\tginv(\NN')$, so by \textbf{TA.1}, there is a transition
$((\kk!\ii.\xx,\NN,\DD),\clkvl) \goes{\kk!\ii.\vv}
((\tick,\NN',\DD),\clkvl[\uclock:=0])$. 
It is easy to see that $\age(\NN',\clkvl[\uclock:=0]) = \N'$, and, therefore,
$((\tick,\NN',\DD),\clkvl[\uclock:=0]) \birel (\tick,\N',\D)$.
\item \textbf{Snd.2}: 
$\any \in \NetLabels$, $\state' \syneq (\kk!\ii.\xx,\N',\D)$. 

There are four sub-cases to consider: one for each of the ways in 
which the network transition can be derived.
\begin{enumerate}
\item \textbf{N.1}: similar to the following case.
\item \textbf{N.2}:
$\any = \kk\offers{\mm}$, $\N_\kk = (\preact{0,\ti}{\mm},\mq)$,
$\N' = \assign{\N}{\kk}{(\offers{\mm},\mq)}$

In this case, since $\age(\NN,\clkvl) = \N$, then, by
Definition~\ref{def:pfage}, $\NN_\kk =
(\preact{\ti_1,\ti_2}{\mm},\mq)^\clock$ and $\clkvl(\clock) \geq
\ti_1$. So by \textbf{E\_Snd.2} and \textbf{E\_N.2}, there is an edge
$(\kk!\ii.\xx,\NN,\DD) \goes{\clock \geq \ti_1,
\kk\offers{\mm},\{\uclock\}} (\kk!\ii.\xx,\NN',\DD)$, where $\NN' =
\assign{\NN}{\kk}{(\offers{\mm},\mq)^\clock}$. Clearly, 
$\clkvl \models \clock \geq \ti_1$, and $\clkvl[\uclock:=0] \models
\tginv(\kk!\ii.\xx,\NN',\DD)$, so by \textbf{TA.1}, there is a
transition $((\kk!\ii.\xx,\NN,\DD),\clkvl) \goes{\kk\offers{\mm}} ((\kk!\ii.\xx,\NN',\DD),\clkvl[\uclock:=0])$. Obviously, $((\kk!\ii.\xx,\NN',\DD),\clkvl[\uclock:=0]) \birel (\kk!\ii.\xx,\N',\D)$.
\item \textbf{N.3}: similar to the previous case. 
\item \textbf{N.4}: similar to the previous case. 
\end{enumerate}
\item \textbf{Snd.3}:
$\any = 0$, $\state' \syneq (\kk!\ii.\xx,\N,\D)$.

Since $\clkvl \models \tginv(\kk!\ii.\xx,\NN,\DD)$, then by
\textbf{TA.2}, $((\kk!\ii.\xx,\NN,\DD),\clkvl) \goes{0}$ \break
$((\kk!\ii.\xx,\NN,\DD),\clkvl)$.  We already have that
$((\kk!\ii.\xx,\NN,\DD),\clkvl) \birel (\kk!\ii.\xx,\N,\D)$.
\end{enumerate}
\item \case $\sstate \syneq ((\kk?\ii.\xx,\NN,\DD),\clkvl)$: similar to the 
  previous case.
\item \case $\sstate \syneq (([\op:\ti_1,\ti_2]^\clock,\NN,\DD),\clkvl)$: 
  similar to the previous case.
\item \case $\sstate \syneq ((\g\guard\P,\NN,\DD),\clkvl)$: similar to the 
  following case.
\item \case $\sstate \syneq ((\PP\sq\QQ,\NN,\DD),\clkvl)$ \\
So, since $\sstate\birel\state$, we have $\state \syneq (\P\sq\Q,\N,\D)$, where
$\P = \age(\PP,\clkvl)$, $\Q = \unclk(\QQ)$, $\N = \age(\NN,\clkvl)$,
and $\D = \DD$.

There are two sub-cases to consider:
\begin{enumerate}
\item \textbf{Seq.1}:
$\any \in \ProgLabels \cup \NetLabels \cup \Time$, $\state' \syneq (\P'\sq\Q,\N',\D')$, $\P'\nsyneq\tick$

By \textbf{Seq.1}, $(\P,\N,\D) \goes{\any} (\P',\N',\D')$, where $\P'
\nsyneq \tick$, and so, by i.h., 
$((\PP,\NN,\DD),\clkvl) \goes{\any} ((\PP',\NN',\DD'),\clkvl')$, 
and there exist $\PP'' \bisim \PP'$ and $\P'' \bisim \P'$ such that 
$((\PP'',\NN',\DD'),\clkvl')\birel(\P'',\N',\D')$. 
There are now two sub-cases to consider:
\begin{enumerate}
\item $\any \in \ProgLabels \cup \NetLabels$:
So $((\PP,\NN,\DD),\clkvl) \goes{\any} ((\PP',\NN',\DD'),\clkvl')$ must be 
derived by \textbf{TA.1} from an edge $(\PP,\NN,\DD) 
\goes{\clkcond,\any,\resets} (\PP',\NN',\DD')$, 
where $\clkvl \models \clkcond$ and 
$\clkvl' = \clkvl[\resets:=0] \models \tginv(\PP',\NN',\DD')$. 
Now, by \textbf{E\_Seq.1}, there must be an
edge $(\PP\sq\QQ,\NN,\DD) \goes{\clkcond,\any,\resets} (\PP'\sq\QQ,\NN',\DD')$.
We already have $\clkvl \models \clkcond$ and, since $\clkvl[\resets:=0]
\models (\PP',\NN',\DD')$, we also have $\clkvl[\resets:=0] \models 
\tginv(\PP'\sq\QQ,\NN',\DD')$, by Definition~\ref{def:tginvariant}. 
So, by TA.1, there is a transition $((\PP\sq\QQ,\NN,\DD),\clkvl)
\goes{\any} ((\PP'\sq\QQ,\NN'\DD'),\clkvl')$. 
Since $((\PP'',\NN',\DD'),\clkvl') \birel$  
$(\P'',\N',\D')$, and $\unclk(\QQ) = \Q$, then  
$((\PP''\sq\QQ,\NN',\DD'),\clkvl') \birel$ $(\P''\sq\Q,\N',\D')$. Moreover, 
$\PP''\sq\QQ \bisim \PP'\sq\QQ$ and $\P''\sq\Q  \bisim \P'\sq\Q$, as required.
\item $\any = \ti \in \Time$: \\
So $((\PP,\NN,\DD),\clkvl) \goes{\ti} ((\PP',\NN',\DD'),\clkvl')$ must
be derived by \textbf{TA.2} with 
$(\PP',\NN',\DD') \syneq (\PP,\NN,\DD)$, $\clkvl' = \clkvl+\ti$,
and $\forall \ti' \in \Time | \ti' \leq \ti
\such \clkvl+\ti' \models \tginv(\PP,\NN,\DD)$. By 
Definition~\ref{def:tginvariant}, we have 
$\forall \ti' \in \Time | \ti' \leq \ti \such \clkvl+\ti' \models 
\tginv(\PP\sq\QQ,\NN,\DD)$, and so by \textbf{TA.2},
there is a transition $((\PP\sq\QQ,\NN,\DD),\clkvl) \goes{\ti} ((\PP'\sq\QQ,\NN,\DD),\clkvl')$. From i.h., we have  
$\PP'' \bisim \PP'$ and $\P'' \bisim \P'$ such that 
$((\PP'',\NN,\DD),\clkvl') \birel$ $(\P'',\N',\D')$, and so, since
$\unclk(\QQ) = \Q$, we have $((\PP''\sq\QQ,\NN,\DD),\clkvl') \birel
(\P''\sq\Q,\N',\D')$,
where $\PP''\sq\QQ \bisim \PP'\sq\QQ$  and $\P''\sq\Q \bisim \P'\sq\Q$.
 \end{enumerate}
\item \textbf{Seq.2}:
$\any = \prog \in \ProgLabels$, $\state' \syneq (\Q,\N',\D')$

By \textbf{Seq.2}, $(\P,\N,\D) \goes{\prog} (\tick,\N',\D')$, and so, 
by i.h.,  
$((\PP,\NN,\DD),\clkvl)$ $\goes{\prog}((\PP',\NN',\DD'),\clkvl')$, where 
there exist $\PP'' \bisim \PP'$  and $\P'' \bisim \tick$, such that 
$((\PP'',\NN',\DD'),\clkvl')\birel(\P'',\N',\D')$. But, by definition of
$\age$ and Proposition~\ref{prop:pftickdistinguished}, we must 
therefore have $\PP'' \syneq \PP' \syneq \P'' \syneq \tick$.

The transition $((\PP,\NN,\DD),\clkvl) \goes{\prog} 
((\tick,\NN',\DD'),\clkvl')$
must be derived by \textbf{TA.1} from an edge $(\PP,\NN,\DD)
\goes{\clkcond,\prog,\resets} (\tick,\NN',\DD')$, where $\clkvl
\models \clkcond$ and $\clkvl' = \clkvl[\resets:=0]
\models \tginv(\tick,\NN',\DD')$. So, by \textbf{E\_Seq.2}, there is an
edge $(\PP\sq\QQ,\NN,\DD)
\goes{\clkcond,\prog,\resets\cup\iclk(\QQ)} (\QQ,\NN',\DD')$.  We
already have $\clkvl \models \clkcond$ and, by Lemmas~\ref{lem:pfreset}
and~\ref{lem:pfprocinv}, we have $\clkvl[\resets\cup\iclk(\QQ):=0]
\models (\QQ,\NN',\DD')$. 
So, by TA.1, there is a transition  $((\PP\sq\QQ,\NN,\DD),\clkvl)
\goes{\prog} ((\QQ,\NN',\DD'),\clkvl'')$, where $\clkvl'' = 
\clkvl[\resets\cup\iclk(\QQ):=0]$. 
Since $((\tick,\NN',\DD'),\clkvl')\birel(\tick,\N',\D')$, then,
$\D' = \DD'$, and, by Lemma~\ref{lem:pfage}, $\N' = \age(\NN',\clkvl'')$. 
Let $\Qq = \age(\QQ,\clkvl'')$.
Clearly, $((\QQ,\NN',\DD'),\clkvl'')\birel(\Qq,\N',\D')$.
But, $\Q = \unclk(\QQ)$, and
by Lemma~\ref{lem:pfprocage}, $\unclk(\QQ) \bisim \Qq$, 
so $\Q \bisim \Qq$, as required. 
\end{enumerate}

\item \case $\sstate \syneq ((\PP\choice\QQ,\NN,\DD),\clkvl)$: similar to the 
  following case.
\item \case $\sstate \syneq ((\PP\interrupt\QQ,\NN,\DD),\clkvl)$. \\
So, since $\sstate\birel\state$, we have 
$\state \syneq (\P\interrupt\Q,\N,\D)$, where 
$\age(\PP,\clkvl) = \P$, $\age(\QQ,\clkvl) = \Q$, 
$\age(\NN,\clkvl) = \N$ and $\DD = \D$.

There are four sub-cases to consider:
\begin{enumerate}
\item \textbf{Int.1}: $\any = \prog \in \ProgLabels$,
$\state' \syneq (\P'\interrupt\Q,\N',\D')$, $\P' \nsyneq \tick$.

By \textbf{Int.1}, $(\P,\N,\D) \goes{\prog} (\P',\N',\D')$ and
so, by i.h., $((\PP,\NN,\DD),\clkvl)$ $\goes{\prog}
((\PP',\NN',\DD'),\clkvl')$, and there exist $\PP'' \bisim \PP'$ and 
$\P'' \bisim \P'$ such that
$((\PP'',\NN',\DD'),\clkvl')\birel(\P'',\N',\D')$. But this transition
must be derived by \textbf{TA.1} from an edge $(\PP,\NN,\DD)
\goes{\clkcond,\prog,\resets} (\PP',\NN',\DD')$, where $\PP' \nsyneq
\tick$, $\clkvl \models \clkcond$ and $\clkvl' = \clkvl[\resets:=0]
\models \tginv(\PP',\NN',\DD')$. So, by \textbf{E\_Int.1}, there is
an edge $(\PP\interrupt\QQ,\NN,\DD) \goes{\clkcond,\prog,\resets}(\PP'\interrupt\QQ,\NN',\DD')$. We already have $\clkvl[\resets:=0] 
\models \tginv(\PP',\NN',\DD')$ and $\clkvl \models \tginv(\QQ,\NN,\DD)$.
Since the urgent clock $\uclock$ is reset by every edge, then, by 
Lemmas~\ref{lem:pfreset} and~\ref{lem:pfinvpreservedinnewenv}, we have
$\clkvl[\resets:=0] \models \tginv(\QQ,\DD')$. So, by
Definition~\ref{def:tginvariant}, we have
$\clkvl[\resets:=0] \models
\tginv(\PP'\interrupt\QQ,\NN',\DD')$. Moreover, $\clkvl \models \clkcond$, 
so by \textbf{TA.1}, there is a transition
$((\PP\interrupt\QQ,\NN,\DD),\clkvl) \goes{\prog}
((\PP'\interrupt\QQ,\NN',\DD'),\clkvl[\resets:=0])$.

Since $\PP'' \bisim \PP'$ and $\P'' \bisim \P'$, then 
$\PP''\interrupt\QQ \bisim \PP'\interrupt\QQ$ and
$\P''\interrupt\Q \bisim \P'\interrupt\Q$.
To see that
$((\PP''\interrupt\QQ,\NN',\DD'),\clkvl[\resets:=0])\birel
(\P''\interrupt\Q,\N',\D')$, we reason as follows. By construction of 
$\birel$, $\P'' = \age(\PP'',\clkvl[\resets:=0])$. Furthermore, by safety of 
clock variable allocation, $\resets \cap \iclk(\QQ) = \emptyset$, so, by
Lemma~\ref{lem:pfage}, $\Q = \age(\QQ,\clkvl[\resets:=0])$ and $\N' =
\age(\NN',\clkvl[\resets:=0])$. The result follows by construction of
$\birel$. 
\item \textbf{Int.2}: similar to previous case.
\item \textbf{Int.3}: similar to previous case.
\item \textbf{Int.4}: $\any = \nort \in \NetLabels \cup \Time$,
$\state' \syneq (\P'\interrupt\Q',\N',\D')$ \\
There are two sub-cases to consider:
\begin{enumerate}
\item $\any = \netw \in \NetLabels$: \\ 
In this case, by Lemma~\ref{lem:pfnetdeterm}, we have 
$\P'\interrupt\Q' \syneq \P\interrupt\Q$ and $\D' = \D$.
So, by \textbf{Int.4}, 
$(\P,\N,\D) \goes{\netw} (\P,\N',\D)$ and 
$(\Q,\N,\D) \goes{\netw} (\Q,\N',\D)$.
Moreover, by i.h. and Lemma~\ref{lem:pfcnetdeterm}, 
$((\PP,\NN,\DD),\clkvl) \goes{\netw} ((\PP,\NN',\DD),\clkvl')$ and
$(\QQ,\NN,\DD),\clkvl) \goes{\netw}$ \break $((\QQ,\NN',\DD),\clkvl')$, and
there exist $\PP'' \bisim \PP$, $\QQ'' \bisim \QQ$, $\P''\bisim\P$ and 
$\Q''\bisim\Q$ such that
$((\PP'',\NN',\DD),\clkvl')\birel(\P'',\N',\D)$ and 
$((\QQ'',\NN',\DD),\clkvl')\birel(\Q'',\N',\D)$.

But
$((\PP,\NN,\DD),\clkvl) \goes{\netw} ((\PP,\NN',\DD),\clkvl')$, must be 
derived by \textbf{TA.1} from an edge
$(\PP,\NN,\DD) \goes{\clkcond,\netw,\resets} (\PP,\NN',\DD)$, where 
$\clkvl \models \clkcond$ and $\clkvl' = \clkvl[\resets:=0] \models
\tginv(\PP,\NN',\DD)$. Similarly for $((\QQ,\NN,\DD),\clkvl)$ $\goes{\netw} 
((\QQ,\NN',\DD),\clkvl')$ --- Lemma~\ref{lem:pfcnetdeterm}
ensures that this edge will have the same clock guard and reset set as the edge
for $(\PP,\NN,\DD)$. So by \textbf{E\_Int.4}, there is an edge
$(\PP\interrupt\QQ,\NN,\DD) \goes{\clkcond,\netw,\resets}$
$(\PP\interrupt\QQ,\NN',\DD)$, and, since $\clkvl \models \clkcond$ and
$\clkvl' \models \tginv(\PP\interrupt\QQ,\NN',\DD)$, there is
a transition $((\PP\interrupt\QQ,\NN,\DD),\clkvl) \goes{\netw}
((\PP\interrupt\QQ,\NN',\DD),\clkvl')$. Clearly, by 
i.h. and Definition~\ref{def:pfage}, 
$((\PP''\interrupt\QQ'',\NN',\DD),\clkvl')
\birel(\P''\interrupt\Q'',\N',\D)$, where
$\PP'' \interrupt \QQ'' \bisim \PP \interrupt \QQ$ and
$\P''\interrupt\Q'' \bisim \P \interrupt \Q$.

\item $\any = \ti \in \Time$: \\
In this case, by Lemma~\ref{lem:pfnetdeterm}, we have $\D' = \D$. 
So, by \textbf{Int.4}, 
$(\P,\N,\D) \goes{\ti} (\P',\N',\D)$ and 
$(\Q,\N,\D) \goes{\ti} (\Q',\N',\D)$. 
Moreover, by i.h., 
$((\PP,\NN,\DD),\clkvl) \goes{\ti} ((\PP',\NN',\DD'),\clkvl')$ and 
$(\QQ,\NN,\DD),\clkvl)$ $\goes{\ti} ((\QQ',\NN'',\DD''),\clkvl'')$, and
there exist $\PP'' \bisim \PP$, $\QQ'' \bisim \QQ$, $\P''\bisim\P$ and 
$\Q''\bisim\Q$ such that
$((\PP'',\NN',\DD'),\clkvl')\birel(\P'',\N',\D)$ and 
$((\QQ'',\NN'',\DD''),\clkvl'')\birel(\Q'',\N',\D)$. 
But, $((\PP,\NN,\DD),\clkvl) \goes{\ti}$
$((\PP,\NN,\DD),\clkvl')$, must be derived by \textbf{TA.2},
where $(\PP',\NN',\DD') \syneq (\PP,\NN,\DD)$, $\clkvl' = \clkvl +\ti$ and 
$\forall \ti' \in \Time | \ti' \leq \ti \such \clkvl+\ti' \models
\tginv(\PP,\NN,\DD)$. Similarly for $\QQ$. 
So, from Definition~\ref{def:tginvariant},
it is clear that $\forall \ti' \in \Time | \ti' \leq \ti \such
\clkvl+\ti' \models \tginv(\PP\interrupt\QQ,\NN,\DD)$, and, therefore,
by \textbf{TA.2}, $((\PP\interrupt\QQ,\NN,\DD),\clkvl) \goes{\ti}
((\PP\interrupt\QQ,\NN,\DD),\clkvl+\ti)$. By i.h and construction
of $\birel$, $(\PP''\interrupt\QQ'',\NN,\DD),\clkvl+\ti)\birel(\P''\interrupt\Q'',\N',\D)$, where $\PP''\interrupt\QQ'' \bisim \PP\interrupt\QQ$ and
$\P''\interrupt\Q'' \bisim \P'\interrupt\Q'$.
\end{enumerate}
\end{enumerate}

\item \case $\sstate \syneq ((\PP\parallel\QQ,\NN,\DD),\clkvl)$: similar to 
  the previous case. 

\item \case $\sstate \syneq ((\rec X.\PP_1,\NN,\DD),\clkvl)$ \\
So, since $\sstate\birel\state$, we have
$\state \syneq (\P,\N,\D)$, where $\P = \age(\rec X.\PP_1,\clkvl)$,
$\N = \age(\NN,\clkvl)$, and $\D = \DD$. But $\age(\rec X.\PP_1,\clkvl) =$ 
\break $\age(\PP_1[\rec X.\PP_1/X],\clkvl)$, which is derived by a shorter
calculation, and so, if $(\P,\N,\D) \goes{\any} (\P',\N',\D')$, 
then, by i.h., \break 
$((\PP_1[\rec X.\PP_1/X],\NN,\DD),\clkvl) \goes{\any}
((\PP',\NN',\DD'),\clkvl')$ and there exist $\PP'' \bisim \PP'$ and 
$\P'' \bisim \P'$ such that
$((\PP'',\NN',\DD'),\clkvl')\birel(\P'',\N',\D')$.
There are two sub-cases to consider:
\begin{enumerate}
\item $\any \in \ProgLabels \cup \NetLabels$: \\
So
$((\PP_1[\rec X.\PP_1/X],\NN,\DD),\clkvl) \goes{\any}
((\PP',\NN',\DD'),\clkvl')$ must be derived by \textbf{TA.1} from an
edge $(\PP_1[\rec X.\PP_1/X],\NN,\DD) \goes{\clkcond,\any,\resets}
(\PP',\NN',\DD')$, where $\clkvl \models \clkcond$ and $\clkvl' = 
\clkvl[\resets:= 0] \models \tginv(\PP',\NN',\DD')$. But then,
by \textbf{E\_Rec}, there is an edge 
$(\rec X.\PP_1,\NN,\DD) \goes{\clkcond,\any,\resets}
(\PP',\NN',\DD')$, and so, by \textbf{TA.1}, there is a transition 
$((\rec X.\PP_1,\NN,\DD),\clkvl) \goes{\any}$ $((\PP',\NN',\DD'),\clkvl')$. 
By i.h., we have $((\PP'',\NN',\DD'),\clkvl')\birel(\P'',\N',\D')$, where
$\PP'' \bisim \PP'$ and $\P''\bisim\P'$. 
\item $\any = \ti \in \Time$: \\
So
$((\PP_1[\rec X.\PP_1/X],\NN,\DD),\clkvl) \goes{\ti} 
((\PP',\NN',\DD'),\clkvl')$ must be derived by \textbf{TA.2}, 
where $(\PP',\NN',\DD') \syneq (\PP_1[\rec X.\PP_1/X],\NN,\DD)$,
$\clkvl' = \clkvl+\ti$ and $\forall \ti' \in \Time | \ti' \leq \ti 
\such \clkvl + \ti' \models \tginv(\PP',\NN',\DD')$. 
But then, by
Definition~\ref{def:tginvariant}, $\forall \ti' \in \Time | \ti' \leq \ti \such
\clkvl + \ti' \models \tginv(\rec X.\PP_1,\NN',\DD')$, and so, by
\textbf{TA.2}, there is a transition $((\rec X.\PP_1,\NN,\DD),\clkvl)
\goes{\ti}$ $((\rec X.\PP_1,\NN',\DD'),\clkvl')$. 
By i.h., we have $((\PP'',\NN',\DD'),\clkvl')\birel$ \break 
$(\P'',\N',\D')$, where 
$\PP'' \bisim \PP' \syneq \PP_1[\rec X.\PP_1/X] \bisim \rec X.\PP_1$, 
and $\P'' \bisim \P'$, as required. 
\end{enumerate} 
\end{enumerate}
\item 
  It is enough to show 
  that for all $\any \in \Labels$, 
  if $\sstate = ((\PP,\NN,\DD),\clkvl) \goes{\any}$ \break
  $((\PP',\NN',\DD'),\clkvl')$,
  then $\state = (\P,\N,\D) \goes{\any} (\P',\N',\D')$, 
  and there exist $\PP'' \bisim \PP'$ and $\P'' \bisim \P'$ such that  
  $((\PP'',\NN',\DD'),\clkvl') \birel (\P'',\N',\D')$.
  Again, the proof is by induction on the number of steps in the
  calculation of $\age(\sstate)$. 
  The proof is symmetrical to the previous case. We provide some illustrative 
  variations.
\begin{enumerate}
\item \case $\sstate \syneq (([\op:\ti_1,\ti_2]^\clock,\NN,\DD),\clkvl)$ \\
  Since $\sstate\birel\state$, then $\state \syneq
  (\P,\N,\D)$, where $\P = \age(\PP,\clkvl)$, $\N = \age(\NN,\clkvl)$ and
  $\D = \DD$. There are three sub-cases to consider:
  \begin{enumerate}
    \item $\any = \op \in \ProgLabels$ \\
      A transition $(([\op:\ti_1,\ti_2]^\clock,\NN,\DD),\clkvl) \goes{\op} 
      ((\PP',\NN',\DD'),\clkvl')$ must be derived by \textbf{TA.1} from an edge
      $([\op:\ti_1,\ti_2]^\clock,\NN,\DD) \goes{\clkcond,\op,\resets}
      (\PP',\NN',\DD')$, where $\clkvl \models \clkcond$ and $\clkvl' = 
      \clkvl[\resets:=0] \models \tginv(\PP',\NN',\DD')$. 
      Such an edge can be constructed only by
      \textbf{E\_Comp.1}, so $\ti_1 \in \nat$, $\clkcond \syneq \clock \geq 
      \ti_1$, $\resets = \{\uclock\}$, $\PP' = \tick$, $\NN' = \NN$ and
      $\DD \goesdata{\op} \DD'$. Since $\clkvl \models \clock \geq \ti_1$
      and $\clkvl \models \clock \leq \ti_2$, 
      then $\ti_2 \geq \clkvl(\clock) \geq \ti_1$ and so 
      $\P = \age([\op:\ti_1,\ti_2]^\clock,\clkvl) = 
      [\op:0,\ti_2\nmin\clkvl(\clock)]$.
      Also, $\D = \DD \goesdata{\op} \DD'$. So by \textbf{Comp.1},
      $(\P,\N,\D) \goes{\op} (\tick,\N,\D')$ where $\D' = \DD'$. Clearly,
      $((\tick,\NN,\DD'),\clkvl')\birel(\tick,\N,\D')$.
    \item $\any = \netw \in \NetLabels$ \\
      A transition $(([\op:\ti_1,\ti_2]^\clock,\NN,\DD),\clkvl) \goes{\netw} 
      ((\PP',\NN',\DD'),\clkvl')$ must be derived by \textbf{TA.1} from an edge
      $([\op:\ti_1,\ti_2]^\clock,\NN,\DD) \goes{\clkcond,\netw,\resets}
      (\PP',\NN',\DD')$, where $\clkvl \models \clkcond$ and $\clkvl' = 
      \clkvl[\resets:=0] \models \tginv(\PP',\NN',\DD')$. Such an edge can be
      constructed only by \textbf{E\_Comp.2}, so $\PP' = 
      \PP$, $\DD' = \DD$ and $\NN \goesn{\netw} \NN'$.
      There are four sub-cases to consider: one for each of the ways in which 
      $\NN \goesn{\netw} \NN'$ can be derived. We show the case 
      \textbf{E\_N.4}. Cases \textbf{E\_N.1} -- \textbf{E\_N.3} can be proved 
      similarly.
      \begin{enumerate}
        \item \textbf{E\_N.4}. \\
          If $\NN \goesn{\netw} \NN'$ is derived by \textbf{E\_N.4},
          then, for some channel identifier $\kk \in \K$,
          $\NN_\kk = (\postact{\tlb,\tub}{\mm},\mq)^{\clock_\kk}$, 
          $\NN' = \assign{\NN}{\kk}{(\free,\mq)^{\clock_\kk}}$, 
          $\clkcond \syneq \clock_\kk \geq \tlb$, $\netw = \kk\free$
          and $\resets = \{\uclock\}$. We have $\clkvl \models
          \clock_\kk \leq \tub$ and $\clkvl \models \clock_\kk \geq \tlb$,
          so $\tlb \leq \clkvl(\clock_\kk) \leq \tub$, and therefore
          $\N_\kk = \age(\NN_\kk, \clkvl) = (\postact{0,\tub\nmin
          \clkvl(\clock_\kk)}{\mm},\mq)$. It follows, by \textbf{N.4},
          that $\N \goesn{\kk\free} \N'$, where $\N' = \assign{\N}{\kk} 
          {(\free,\mq)}$. So, by \textbf{Comp.2}, we have
          $(\P,\N,\D) \goes{\kk\free} (\P,\N',\D)$. To show that
          $((\PP,\NN',\DD),\clkvl')\birel(\P,\N',\D)$, we reason as follows.
          Since $\resets = \{\uclock\}$, then $\clkvl' = 
          \clkvl[\uclock:=0]$. Now, for the process term, we have 
          $\P = \age([\op:\ti_1,\ti_2]^\clock,\clkvl)
          = \age([\op:\ti_1,\ti_2]^\clock,\clkvl')$, since, by safety 
          of clock variable allocation, $\clock \neq \uclock$. 
          For the network, we
          have $\NN' = \assign{\NN}{\kk}{(\free, \mq)^{\clock_\kk}}$. 
          But $\age(\NN,\clkvl) = \N$ and 
          $\age((\free,\mq)^{\clock_\kk},\clkvl') =
          (\free,\mq)$, so, by safety of clock variable allocation,
          $\age(\NN',\clkvl') = \N'$. Finally, $\D=\DD$.
          The result follows.
      \end{enumerate}
    \item $\any = \ti \in \Time$ \\
      A transition $(([\op:\ti_1,\ti_2]^\clock,\NN,\DD),\clkvl) \goes{\ti} 
      ((\PP',\NN',\DD'),\clkvl')$ must be derived by \textbf{TA.2}, so
      $(\PP',\NN'\DD') = ([\op:\ti_1,\ti_2]^\clock,\NN,\DD)$, $\clkvl' = 
      \clkvl + \ti$, and $\forall \ti' \in \Time | \ti' \leq \ti \such \clkvl
      + \ti' \models \tginv([\op:\ti_1,\ti_2]^\clock,\NN,\DD)$. By 
      Definition~\ref{def:tginvariant}, $\clkvl + \ti \models 
      \clock \leq \ti_2 \land \tginv(\NN)$. Now, we have $\P = 
      \age([\op:\ti_1,\ti_2]^\clock,\clkvl) = [\op: \ti_1\nmin\clkvl(\clock),
      \ti_2\nmin\clkvl(\clock)]$, and, since $\clkvl(\clock) + \ti \leq \ti_2$,
      then $\ti \leq \ti_2 - \clkvl(\clock)$. Also, we have $\N = 
      \age(\NN,\clkvl)$, and, by Lemma~\ref{lem:pfnetprogress}, 
      $\ti \leq \tcp(\N)$, so, by
      \textbf{N.4}, we have $\N \goesn{\ti} \N'$, where $\N' = \N + \ti$.
      Therefore, by \textbf{Comp.3}, we derive 
      $(\P,\N,\D) \goes{\ti} (\P',\N',\D')$, where $\P' = 
     [\op: \ti_1\nmin\clkvl(\clock)\nmin\ti,\ti_2\nmin\clkvl(\clock)\nmin\ti]$,
      and $\D' = \D$. It is a simple application of the definitions
      to show that $((\PP',\NN',\DD'),\clkvl')\birel(\P',\N',\D')$.
  \end{enumerate}
\item \case $\sstate \syneq ((\PP\interrupt\QQ,\NN,\DD),\clkvl)$. \\
  Since $\sstate\birel\state$, then $\state \syneq (\P\interrupt\Q,\N,\D)$, where
  $\P = \age(\PP,\clkvl)$, $\Q = \age(\QQ,\clkvl)$, $\N = \age(\NN,\clkvl)$
  and $\D = \DD$. There are five sub-cases to consider: four
  for each of the ways in which an edge can be constructed which justifies
  a discrete transition by \textbf{TA.1}, and one for the justification of a 
  time transition by \textbf{TA.2}.
  \begin{enumerate}
    \item \textbf{E\_Int.1} \\
      In this case, the edge is of the form $(\PP\interrupt\QQ,\NN,\DD)
      \goes{\clkcond,\prog,\resets}
      (\PP'\interrupt\QQ,\NN',\DD')$, with $\any = \prog \in \Actions$, 
      $\PP' \nsyneq \tick$, $\clkvl \models \clkcond$, and
      $\clkvl[\resets:=0] \models 
      \tginv(\PP'\interrupt\QQ,\NN',\DD')$. But this edge must be derived
      from an edge $(\PP,\NN,\DD) \goes{\clkcond,\prog,\resets}$ 
      $(\PP',\NN',\DD')$. Now, by \textbf{TA.1}, this edge justifies a 
      transition $((\PP,\NN,\DD),\clkvl) \goes{\prog}$ $((\PP',\NN',\DD'),
      \clkvl[\resets:=0])$. So, by i.h., there is a transition
      $(\P,\N,\D) \goes{\prog}$ $(\P',\N',\D')$, where there exist
      $\PP'' \bisim \PP'$ and $\P'' \bisim \P'$ such that
      $((\PP'',\NN',\DD'),\clkvl[\resets:=0])\birel(\P'',\N',\D')$. 
      Since $\PP'\nsyneq\tick$ then, by Definition~\ref{def:pfage} and
      Proposition~\ref{prop:pftickdistinguished}, we have $\P' \nsyneq \tick$,
      and so
      it follows by \textbf{Int.1} that there is a transition
      $(\P\interrupt\Q,\N,\D) \goes{\prog} (\P'\interrupt\Q,\N',\D')$. 
      Since $\PP'' \bisim \PP'$ and $\P'' \bisim \P'$, then 
      $\PP''\interrupt\QQ \bisim \PP'\interrupt\QQ$ and
      $\P''\interrupt\Q \bisim \P'\interrupt\Q$.
      To see that
      $((\PP''\interrupt\QQ,\NN',\DD'),\clkvl[\resets:=0])\birel
      (\P''\interrupt\Q,\N',\D')$, we reason as follows. By construction of 
      $\birel$, $\P'' = \age(\PP'',\clkvl[\resets:=0])$. 
      Furthermore, by safety of 
      clock variable allocation, $\resets \cap \iclk(\QQ) = \emptyset$, so, by
      Lemma~\ref{lem:pfage}, $\Q = \age(\QQ,\clkvl[\resets:=0])$ and 
      $\N' = \age(\NN',\clkvl[\resets:=0])$. 
      The result follows by construction of $\birel$. 
    \item \textbf{E\_Int.2} Similar to previous case.       
    \item \textbf{E\_Int.3} Similar to previous case.
    \item \textbf{E\_Int.4} \\
      In this case the edge is of the form $(\PP\interrupt\QQ,\NN,\DD)
      \goes{\clkcond,\netw,\resets} (\PP\interrupt\QQ,\NN',\DD)$, where
      $\any = \netw \in \NetLabels$, $\clkvl \models \clkcond$, and
      $\clkvl' = \clkvl[\resets:=0] \models \tginv(\PP\interrupt\QQ,\NN',\DD)$.
      But the existence of this edge depends upon edges $(\PP,\NN,\DD)
      \goes{\clkcond,\netw,\resets} (\PP,\NN',\DD)$ and $(\QQ,\NN,\DD)
      \goes{\clkcond,\netw,\resets} (\QQ,\NN',\DD)$. Now, by \textbf{TA.1},
      these edges justify transitions  $((\PP,\NN,\DD),\clkvl) \goes{\netw}$ 
      $((\PP,\NN',\DD),\clkvl')$ and $((\PP,\NN,\DD),
      \clkvl) \goes{\netw} ((\PP,\NN',\DD),\clkvl')$. So, by
      i.h. and Lemma~\ref{lem:pfnetdeterm}, there are transitions 
      $(\P,\N,\D) \goes{\netw} (\P,\N',\D)$ and 
      $(\Q,\N,\D)$ $\goes{\netw} (\Q,\N',\D)$, where
      there exist $\PP''\bisim\PP$, $\P''\bisim\P$, $\QQ''\bisim\QQ$, and
      $\Q''\bisim\Q$ such that
      $((\PP'',\NN',\DD),\clkvl')\birel(\P'',\N',\D)$ and 
      $((\QQ'',\NN',\DD),\clkvl')\birel(\Q'',\N',\D)$.
      Therefore, by \textbf{Int.4},  
      $(\P\interrupt\Q,\N,\D) \goes{\netw} (\P\interrupt\Q,\N',\D)$.
      Clearly, by i.h. and Definition~\ref{def:pfage}, 
      $((\PP''\interrupt\QQ'',\NN',\DD),\clkvl')
      \birel(\P''\interrupt\Q'',\N',\D)$, where
      $\PP'' \interrupt \QQ'' \bisim \PP \interrupt \QQ$ and
      $\P''\interrupt\Q'' \bisim \P \interrupt \Q$.
    \item $\any = \ti \in \Time$ \\
      This transition is derived by \textbf{TA.2}, where 
      $((\PP\interrupt\QQ,\NN,\DD),\clkvl) \goes{\ti}$
      $((\PP,\NN,\DD),\clkvl')$, where $\clkvl' = \clkvl+\ti$. 
      The transition is possible only if
      $\forall \ti' \in \Time | \ti' \leq \ti \such \clkvl + \ti' \models 
      \tginv(\PP\interrupt\QQ,\NN,\DD)$. So, by 
      Definition~\ref{def:tginvariant}, $\clkvl + \ti \models 
      \tginv(\PP,\DD) \land \tginv(\QQ,\DD) \land \tginv(\NN)$, and, therefore,
      by \textbf{TA.2}, $((\PP,\NN,\DD),\clkvl) \goes{\ti} 
      ((\PP,\NN,\DD),\clkvl')$, and  $((\QQ,\NN,\DD),\clkvl) \goes{\ti}$ 
      $((\QQ,\NN,\DD),\clkvl')$. By i.h. and Lemma~\ref{lem:pfnetdeterm},  
      $(\P,\N,\D) \goes{\ti} (\P',\N',\D)$ and $(\Q,\N,\D) \goes{\ti} 
      (\Q',\N',\D)$
      and there exist $\PP''\bisim\PP$, $\P''\bisim\P'$, $\QQ''\bisim\QQ$
      and $\Q''\bisim\Q'$ such that
      $((\PP'',\NN,\DD),\clkvl')\birel(\P',\N',\D)$ and
      $((\QQ'',\NN,\DD),\clkvl')\birel(\Q',\N',\D)$. Therefore,
      by \textbf{Int.4}, $(\P\interrupt\Q,\N,\D) \goes{\ti}
      (\P'\interrupt\Q',\N',\D)$, and, by construction of $\birel$,
      $((\PP''\interrupt\QQ'',\NN,\DD),\clkvl')\birel
       (\P''\interrupt\Q'',\N',\D)$, where $\PP''\interrupt\QQ''\bisim
       \PP\interrupt\QQ$ and $\P''\interrupt\Q'' \bisim \P'\interrupt\Q'$.
  \end{enumerate}  
\item \case $\sstate \syneq ((\rec X.\PP_1,\NN,\DD),\clkvl)$ \\
So, since $\sstate\birel\state$, we have
$\state \syneq (\P,\N,\D)$, where $\P = \age(\rec X.\PP_1,\clkvl)$,
$\N = \age(\NN,\clkvl)$, and $\D = \DD$. 
There are two sub-cases to consider:
\begin{enumerate}
\item $\any \in \ProgLabels \cup \NetLabels$: \\
Then, the transition $((\rec X.\PP_1,\NN,\DD),\clkvl) \goes{\any}
((\PP',\NN',\DD'),\clkvl')$ must be derived by \textbf{TA.1} from
an edge $(\rec X.\PP_1,\NN,\DD) \goes{\clkcond,\any,\resets}
(\PP',\NN',\DD')$, where $\clkvl \models \clkcond$ and
$\clkvl' = \clkvl[\resets:=0] \models \tginv(\PP',\NN',\DD')$. This edge
must be justified by \textbf{E\_Rec} from an edge
$(\PP_1[\rec X.\PP_1/X],\NN,\DD)$ $\goes{\clkcond,\any,\resets}
(\PP',\NN',\DD')$. So, by \textbf{TA.1}, there is a transition \break
$((\PP_1[\rec X.\PP_1/X],\NN,\DD),\clkvl) \goes{\any}
((\PP',\NN',\DD'),\clkvl')$. But $\age(\rec X.\PP_1,\clkvl)$ 
$= \age(\PP_1[\rec X.\PP_1/X],\clkvl)$, which 
is derived by a shorter calculation, and so, by i.h., we
have $(\P,\N,\D) \goes{\any} (\P',\N',\D')$ and there exist 
$\PP'' \bisim \PP'$ and 
$\P'' \bisim \P'$ such that
$((\PP'',\NN',\DD'),\clkvl')\birel(\P'',\N',\D')$.

\item $\any = \ti \in \Time$: \\
Then, the transition $((\rec X.\PP_1,\NN,\DD),\clkvl) \goes{\ti}
((\PP',\NN',\DD'),\clkvl')$ must be derived by \textbf{TA.2},
where $(\PP',\NN',\DD') \syneq (\rec X.\PP_1,\NN,\DD)$,
$\clkvl' = \clkvl + \ti$ and $\forall \ti' \in \Time | \ti' \leq \ti \such 
\clkvl + \ti' \models \tginv(\rec X.\PP_1,\NN,\DD)$. But then, by 
Definition~\ref{def:tginvariant}, we have $\forall \ti' \in \Time | 
\ti' \leq \ti \such  \clkvl + \ti'
\models \tginv(\PP_1[\rec X.\PP_1/X],\NN,\DD)$, and so, by \textbf{TA.2},
there is a transition $((\PP_1[\rec X.\PP_1/X],\NN,\DD),\clkvl) \goes{\ti}
((\PP_1[\rec X.\PP_1/X],\NN,\DD),\clkvl')$. By i.h., we have
$(\P,\N,\D) \goes{\ti} (\P',\N',\D')$ and there exist 
$\PP'' \bisim \PP_1[\rec X.\PP_1/X]$ and 
$\P'' \bisim \P'$ such that
$((\PP_1[\rec X.\PP_1/X],\NN,\DD),\clkvl')$ $\birel(\P'',\N',\D')$.
But $\PP'' \bisim \PP_1[\rec X.\PP_1/X] \bisim \rec X.\PP_1$, as required.
\end{enumerate}
\end{enumerate}
\end{enumerate}
\qed

















