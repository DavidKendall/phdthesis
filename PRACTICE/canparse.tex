\chapter{The \candle\ Grammar} \label{chap:canparse}

%token            CAN_AND
%token            CAN_ASSIGN
%token            CAN_ASTERISK
%token            CAN_BEHAVIOUR
%token <bool>     CAN_BOOL 
%token            CAN_CHANNEL
%token            CAN_CLOSEPAREN
%token            CAN_CLOSESQB
%token            CAN_COLON
%token            CAN_COLONCOLON
%token            CAN_COMMA
%token            CAN_CONSTANT
%token            CAN_DO
%token            CAN_DOT
%token            CAN_ELAPSE
%token            CAN_ELSE
%token            CAN_ELSIF
%token            CAN_END
%token            CAN_EOF
%token            CAN_EQ
%token            CAN_EXCEPTION
%token            CAN_EXIT
%token            CAN_EVERY
%token            CAN_FORWARDSLASH
%token            CAN_FUNCTION
%token            CAN_GEQ
%token            CAN_GTR
%token            CAN_IDLE
%token            CAN_IF
%token            CAN_IN
%token            CAN_INOUT
%token <int>      CAN_INTEGER
%token            CAN_IS
%token            CAN_LE
%token            CAN_LEQ
%token            CAN_LOOP
%token            CAN_MINUS
%token            CAN_MOD
%token            CAN_MODULE
%token            CAN_NEQ
%token            CAN_NOT
%token            CAN_NULL
%token            CAN_OPENPAREN
%token            CAN_OPENSQB
%token            CAN_OR
%token            CAN_OUT
%token            CAN_PAR
%token            CAN_PLUS
%token            CAN_PROCEDURE
%token            CAN_QUESTION
%token            CAN_RCV
%token            CAN_RETURN
%token            CAN_RIGHTARROW
%token            CAN_SELECT
%token            CAN_SEMICOLON
%token            CAN_SND
%token <Symbol.t> CAN_SYMBOL
%token            CAN_THEN
%token            CAN_TRAP
%token            CAN_TYPE
%token            CAN_UMINUS
%token            CAN_UNITVALUE
%token            CAN_VAR

%left CAN_OR
%left CAN_AND
%nonassoc CAN_NOT
%left CAN_EQ CAN_NEQ CAN_LE CAN_LEQ CAN_GEQ CAN_GTR
%left CAN_PLUS CAN_MINUS
%left CAN_ASTERISK CAN_FORWARDSLASH CAN_MOD
%nonassoc CAN_UMINUS
%start program
%type <Canasta.program> program
%%
%/* parse.mly Rules */

\section{Syntax Notation}
The grammar of \candle\ is described in an extended Backus-Naur-Form (BNF).
\begin{itemize}
\item Italicized words are used to denote syntactic categories 
  (non-terminal symbols), for example:
  \bgrm
  \quad module
  \alt  formalParameter
  \alt  statement
  \egrm
\item {\tt Typewriter} font is used for lexical elements of the
  language (terminal symbols) such as keywords and special symbols,
  for example:
  \bgrm
  \quad \trm{function}
  \alt  \trm{>=}
  \alt  \trm{every}
  \egrm
\item A list of alternative items is written with each alternative
  occurring on a new line, for example:
  \bgrm
     parameterMode \Derive
       \trm{in}
     \alt \trm{out}
     \alt \trm{inout}                
  \egrm
  Indentation is used to show that a new line is intended as a continuation of 
  the previous item, rather than the beginning of a new alternative.
\item \opt{$\cdot$} denotes an optional item, for example:
\bgrm                                    
loopStatement \Derive
  \trm{loop} \opt{\identifier{loop}} \trm{do} seqStatement \trm{end} \trm{loop}
\egrm
is a rule for a \trm{loop} statement, in which a loop{\it Identifier} may be
present but is not required.
\item \zom{$\cdot$} denotes zero or more occurrences of an item, and
  \oom{$\cdot$} denotes one or more occurrences of an item, for example
\bgrm
module \Derive
  \trm{module} \identifier{module} \trm{is} 
  \zom{declSection} \opt{behaviour} \trm{end} \trm{module} 
\egrm 
is a rule which shows that zero or more occurrences of a \emph{declSection}
item may occur in a \emph{module} declaration.
\end{itemize}

\section{Lexical Conventions}
The lexical conventions of \candle\ are standard:
\begin{itemize}
\item Whitespace characters are space, tab and newline. Any string
  starting with two dashes ``{\tt --}'' and ending with a newline is
  a \emph{comment}. Multiple-line comments start with the string ``{\tt (*}''
  and end with the string ``{\tt *)}''.
\item A \emph{number} is any sequence of digits. 
\item An \emph{identifier} is any sequence of characters in the
  set $\{A-Z,a-z,0-9,\_\}$ which starts with a letter, excluding
  the reserved words shown below. All characters
  in an identifier are significant, and case is significant.
\item The \emph{reserved words} are: \\
\begin{tabular}{lllllll}
\trm{and} &
\trm{behaviour} &
\trm{channel}  &
\trm{const} &
\trm{do}  &
\trm{elapse} \\
\trm{else} &
\trm{elsif} &
\trm{end} &
\trm{exception} &
\trm{exit} &
\trm{every} &
\trm{function} \\
\trm{if} &
\trm{idle} &
\trm{in} &
\trm{inout} &
\trm{is} &
\trm{loop} &
\trm{mod} \\
\trm{module} &
\trm{not} &
\trm{null} &
\trm{or} & 
\trm{out} &
\trm{procedure} &
\trm{rcv} \\
\trm{return} &
\trm{select} &
\trm{snd} &
\trm{then} &
\trm{trap} &
\trm{type} &
\trm{var} 
\end{tabular}
\end{itemize}

\section{Modules}
\bgrm
program \Derive \zom{module}                 
\egrm

\bgrm
module \Derive
  \trm{module} \identifier{module} \trm{is} 
  \zom{declSection} \opt{behaviour} \trm{end} \trm{module}
\egrm

\section{Declarations}
\bgrm
declSection \Derive
     typeDeclSection              
\alt constantDeclSection       
\alt variableDeclSection           
\alt functionDeclSection           
\alt procedureDeclSection          
\alt channelDeclSection            
\alt exceptionDeclSection          
\egrm

\subsection{Type Declarations}
\bgrm
typeDeclSection \Derive
  \trm{type} typeDecl \zom{\trm{;} typeDecl}     
\egrm

\bgrm
typeDecl \Derive
  \identifier{type}            
\egrm

\subsection{Constant Declarations}
\bgrm
constantDeclSection \Derive
  \trm{const} constantDecl \zom{\trm{;} constantDecl} 
\egrm

\bgrm
constantDecl \Derive
  \identifier{constant} \trm{:} \identifier{type}
\egrm

\subsection{Variable Declarations}
\bgrm
variableDeclSection \Derive
  \trm{var} variableDecl \zom{\trm{;} variableDecl}
\egrm

\bgrm
variableDecl \Derive
  \identifier{variable} \trm{:} \identifier{type}
\egrm

\subsection{Function and Procedure Declarations}
\bgrm
functionDeclSection \Derive
  \trm{function} functionDecl \zom{\trm{;} functionDecl} 
\egrm

\bgrm
functionDecl \Derive
  \identifier{function} \trm{(} \opt{formalParameter \zom{\trm{;} formalParameter}} \trm{)} 
    \trm{:} \identifier{type}
\egrm

\bgrm
procedureDeclSection \Derive
  \trm{procedure} procedureDecl \zom{\trm{;} procedureDecl} 
\egrm

\bgrm
procedureDecl \Derive
  \identifier{procedure} \trm{(} \opt{formalParameter \zom{\trm{;} formalParameter}} \trm{)} 
\egrm

\bgrm
formalParameter \Derive
  \opt{parameterMode} \identifier{type}
\egrm

\bgrm
parameterMode \Derive
     \trm{in}
\alt \trm{out}
\alt \trm{inout}                
\egrm

\subsection{Channel Declarations}
\bgrm
channelDeclSection \Derive
  \trm{channel} channelDecl \zom{\trm{;} channelDecl} 
\egrm

\bgrm
channelDecl \Derive
  \identifier{channel} \opt{channelSpec}
\egrm

\bgrm
channelSpec \Derive
  \trm{:} \trm{(} messageSpec \zom{\trm{,} messageSpec} \trm{)}
\egrm

\bgrm
messageSpec \Derive
  \identifier{message} \opt{\trm{.} \identifier{type}}
\egrm

\subsection{Exception Declarations}
\bgrm
exceptionDeclSection \Derive
  \trm{exception} exceptionDecl \zom{\trm{;} exceptionDecl}
\egrm

\bgrm
exceptionDecl \Derive
  \identifier{exception} \trm{:} \identifier{type}
\egrm

\section{Expressions}

\bgrm
expression \Derive
     constantLiteral               
\alt identifier                    
\alt \trm{?} \identifier{exception}                               
\alt functionCall               
\alt \trm{-} expression                                 
\alt expression \trm{*} expression                                
\alt expression \trm{/} expression                               
\alt expression \trm{+} expression         
\alt expression \trm{-} expression                                
\alt expression \trm{mod} expression      
\alt expression \trm{=} expression                          
\alt expression \trm{/=} expression                               
\alt expression \trm{<} expression                                
\alt expression \trm{<=} expression                                
\alt expression \trm{>} expression                              
\alt expression \trm{>=} expression                               
\alt \trm{not} expression               
\alt expression \trm{and} expression      
\alt expression \trm{or} expression       
\alt \trm{(} expression \trm{)}
\egrm

The precedence of operators is given below. Operators of equal
precedence are shown on the same line. Operators of lower precedence
are shown first. All operators are left-associative, except unary minus
and logical negation which are non-associative.

\bgrm
\quad\trm{or}
\alt \trm{and}
\alt \trm{not}
\alt \trm{=} \quad \trm{/=} \quad \trm{<} \quad \trm{<=} \quad \trm{>=} \quad \trm{>}
\alt \trm{+} \quad \trm{-}
\alt \trm{*} \quad \trm{/} \quad \trm{mod}
\egrm

\bgrm
constantLiteral \Derive
     \trm{\unitval}
\alt \trm{false}
\alt \trm{true}                 
\alt \numb                   
\egrm

\bgrm
functionCall \Derive
  \identifier{function} \trm{(} \opt{expression \zom{\trm{,} expression}} \trm{)}
\egrm



\section{Behaviour}
\bgrm
behaviour \Derive
  \trm{behaviour} statement      
\egrm

\bgrm
statement \Derive
  seqStatement \trm{|} statement
\alt seqStatement                 
\egrm

\bgrm
seqStatement \Derive
  atomicStatement \trm{;} seqStatement
\alt atomicStatement               
\egrm

\bgrm
atomicStatement \Derive
     \trm{null}               
\alt \trm{idle}
\alt sndStatement               
\alt rcvStatement                  
\alt elapseStatement               
\alt assignmentStatement          
\alt procedureCall                                
\alt ifStatement                   
\alt loopStatement                 
\alt everyStatement                
\alt selectStatement              
\alt trapStatement                 
\alt exitStatement                 
\alt moduleInstantiation          
\egrm

\subsection{Send statement}
\bgrm
sndStatement \Derive
   \trm{snd} \trm{(} \identifier{channel} \trm{,} \identifier{message}
     \opt{\trm{.} expression} \trm{)}
\egrm

\subsection{Receive statement}
\bgrm
rcvStatement \Derive
  \trm{rcv} \trm{(} \identifier{channel} \trm{,} \identifier{message}
    \opt{\trm{.} \identifier{variable}} \trm{)}                                            
\egrm

\subsection{Elapse statement}
\bgrm
elapseStatement \Derive
  \trm{elapse} expression        
\egrm

\subsection{Assignment statement and Procedure Call}
\bgrm
assignmentStatement \Derive
  \identifier{variable} \trm{:=} expression
\egrm

\bgrm
procedureCall \Derive
  \identifier{procedure} \trm{(} \opt{expression \zom{\trm{,} expression}} \trm{)}
\egrm

\subsection{If statement}
\bgrm
ifStatement \Derive
  \trm{if} thenPart \zom{elsifPart} \opt{elsePart} \trm{end} \trm{if}
\egrm

\bgrm
thenPart \Derive
  expression \trm{then} seqStatement   
\egrm

\bgrm
elsifPart \Derive
  \trm{elsif} expression \trm{then} seqStatement
\egrm

\bgrm
elsePart \Derive
  \trm{else} seqStatement       
\egrm

\subsection{Repetition statements}
\bgrm                                    
loopStatement \Derive
  \trm{loop} \opt{\identifier{loop}} \trm{do} seqStatement \trm{end} \trm{loop}
\egrm

\bgrm
everyStatement \Derive
  \trm{every} expression \trm{do} seqStatement \trm{end} \trm{every}
\egrm

\subsection{Select statement}
\bgrm
selectStatement \Derive
  \trm{select} \oom{selectAlternative} \opt{\trm{in} seqStatement} \trm{end} \trm{select}
\egrm

\bgrm
selectAlternative \Derive
  \trm{::} rcvElapseOrModuleInstantiation \opt{\trm{;} seqStatement}  
\egrm

\bgrm
rcvElapseOrModuleInstantiation \Derive
     rcvStatement                 
\alt elapseStatement              
\alt moduleInstantiation          
\egrm

\subsection{Trap statement}
\bgrm
trapStatement \Derive
  \trm{trap} \oom{trapAlternative} \trm{in} seqStatement \trm{end} \trm{trap}
\egrm

\bgrm
trapAlternative \Derive
  \trm{::} \identifier{exception} \trm{=>} seqStatement
\egrm

\bgrm
exitStatement \Derive
  \trm{exit} \identifier{exception} \opt{\trm{(} expression \trm{)}}
\egrm

\subsection{Module Instantiation}
\bgrm 
moduleInstantiation \Derive
  \identifier{module} \opt{\trm{[} renaming \zom{\trm{,} renaming} \trm{]}} 
\egrm

\bgrm
renaming \Derive
  expression \trm{/} identifier
\egrm















