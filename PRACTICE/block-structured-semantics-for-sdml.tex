Recall that in \bcandle\ the primary semantic object in the treatment
of data is a \emph{valuation}, which is simply a mapping from data
variables to values: $\valuation \defs \Var \fun \V$. This simple idea
needs to be extended in two ways in giving the semantics of \sdml:
\begin{enumerate}
\item a standard technique for managing block structure is employed,
  in which a valuation $\val \in \valuation$ is presented as the composition
  of a pair of maps $\store \bcomp \env$, where $\store \in \Store$ is
  a mapping from storage locations to values, $\Store \defs \Location
  \fun \V$, and $\env \in \Env$ is a mapping from data variables to
  storage locations, $\Env \defs \Var \fun \Location$;
\item an auxiliary environment $\Aenv$ is introduced in order to
  keep track of the meaning of the other objects in an \sdml\ program:
  types, constants, functions and procedures.
\end{enumerate}

Now, the semantic functions which are required later are:
\begin{enumerate}
\item a function $\evalCanExpr{e}$ which gives the meaning of 
  an expression $e$: 
  \[ \evalCanExprFN: \canExpr \fun \Aenv \fun \Env \fun \Store \fun \V \]
\item a function $\evalCanSmnt{s}$ which gives the possble meanings of a
  statement $s$:
  \[ \evalCanSmntFN: \canSmnt \fun \Aenv \fun \Env \fun \Store 
     \fun 2^\Store \]
\end{enumerate} 
 
Notice that because \sdml\ is a non-deterministic language,
a statement maps a store to a \emph{set} of result stores. 
Nevertheless, the semantic definition is quite straightforward; the interested
reader is referred to standard texts such as Schmidt~\cite{sch:86}
or Winskel~\cite{win:93} for further details.

In practice, it is convenient to continue to work with simple valuations
when considering the use of \sdml\ with \candle, so we prefer to write
$\evalCanExpr{e}\val$ and $\evalCanSmnt{s}\val$, rather than the more
long-winded $\evalCanExpr{e} \aenv\ \env\ \store$ and 
$\evalCanSmnt{s} \aenv\ \env\ \store$. Firstly, observe  
\begin{itemize}
\item the auxiliary environment $\aenv$ remains constant throughout, so it
 should not cause confusion simply to assume its presence without 
  actually mentioning it;
\item for any valuation $\val \in \valuation$, there exists a store
  $\store \in \Store$ and environment $\env \in \Env$ such that
  $\val = \store \bcomp \env$; similarly, a store and environment define
  a valuation.
\end{itemize}
Now, the following interpretations of $\evalCanExprFN$ and $\evalCanSmntFN$ 
should be clear:
  \[ \evalCanExpr{e}\val = \vv, \]
  where $\vv \in \V$ satisfies 
  \[ \exists \store \in \Store; \env \in \Env \such
  \val = \store \bcomp \env \land \vv = \evalCanExpr{e} \aenv\ \env\ \store\]
and
\begin{zed}
\evalCanSmnt{s}\val = \{ \val' \in \valuation |
      \exists \store,\store' \in \Store; \env \in \Env \such
      \val = \store \bcomp \env \land \\ 
\t3   \store' \in \evalCanSmnt{s} \aenv\
      \env\ \store \land \val' = \store' \bcomp \env \}
\end{zed}
