\chapter{The \sdml\ Grammar} \label{chap:datparse}
%token            CAN_AND
%token            CAN_ANY
%token            CAN_ARRAY
%token            CAN_ASSIGN
%token            CAN_ASTERISK
%token            CAN_BEGIN
%token <bool>     CAN_BOOL 
%token            CAN_BOUNDS
%token            CAN_CLOSECURLY
%token            CAN_CLOSEDSQB
%token            CAN_CLOSEPAREN
%token            CAN_CLOSESQB
%token            CAN_COLON
%token            CAN_COLONCOLON
%token            CAN_COMMA
%token            CAN_CONSTANT
%token            CAN_DATA
%token            CAN_DO
%token            CAN_DOT
%token            CAN_DOTS
%token            CAN_END
%token            CAN_ENUM
%token            CAN_EOF
%token            CAN_EQ
%token            CAN_FI
%token            CAN_FORWARDSLASH
%token            CAN_FUNCTION
%token            CAN_GEQ
%token            CAN_GTR
%token            CAN_IF
%token            CAN_IN
%token            CAN_INFINITY
%token            CAN_IS
%token            CAN_INOUT
%token <int>      CAN_INTEGER
%token            CAN_LE
%token            CAN_LEQ
%token            CAN_MINUS
%token            CAN_MOD
%token            CAN_NEQ
%token            CAN_NOT
%token            CAN_OPENCURLY
%token            CAN_OPENDSQB
%token            CAN_OPENPAREN
%token            CAN_OPENSQB
%token            CAN_OD
%token            CAN_OF
%token            CAN_OR
%token            CAN_OUT
%token            CAN_PLUS
%token            CAN_PROCEDURE
%token            CAN_RECORD
%token            CAN_RETURN
%token            CAN_RIGHTARROW
%token            CAN_SEMICOLON
%token            CAN_SKIP
%token <Symbol.t> CAN_SYMBOL
%token            CAN_TYPE
%token            CAN_UMINUS
%token            CAN_UNITVALUE
%token            CAN_VAR

%left CAN_OR
%left CAN_AND
%nonassoc CAN_NOT
%left CAN_EQ CAN_NEQ CAN_LE CAN_LEQ CAN_GEQ CAN_GTR
%left CAN_PLUS CAN_MINUS
%left CAN_ASTERISK CAN_FORWARDSLASH CAN_MOD
%nonassoc CAN_UMINUS
%start program
%type <Datasta.program> program
%%

\section{Introduction}
\sdml\ is a simple data modelling language which can be used with
\candle\ for describing the data objects and operations in CAN-based systems. 

The notation used in giving the grammar of \sdml\ is the same as the notation
of Appendix~\ref{chap:canparse}, as are the lexical conventions, except for the
reserved words, which are as follows: \\
\begin{tabular}{lllllll}
\trm{and} &
\trm{any} &
\trm{array} &
\trm{begin} &
\trm{boolean} & 
\trm{bounds} &
\trm{const} \\
\trm{data} &
\trm{do} &
\trm{end} &
\trm{fi} &
\trm{function} &
\trm{if} &
\trm{in} \\
\trm{is} &
\trm{inout} &
\trm{mod} &
\trm{not} &
\trm{od} &
\trm{of} &
\trm{or} \\
\trm{out} &
\trm{procedure} &
\trm{return} &
\trm{skip} &
\trm{type} &
\trm{uvalue} &
\trm{var}
\end{tabular}
 
\section{Data Modules}
\bgrm
program \Derive \zom{dataModule}
\egrm

\bgrm 
dataModule \Derive
  \trm{data} \identifier{dataModule} \trm{is} \zom{declSection} \trm{end} \trm{data}
\egrm

\section{Declarations}
\bgrm
declSection \Derive
     typeDeclSection              
\alt constantDeclSection       
\alt functionDeclSection       
\alt procedureDeclSection      
\egrm

\subsection{Type Declarations}
\bgrm
typeDeclSection \Derive
  \trm{type} typeDecl \zom{\trm{;} typeDecl}         
\egrm

\bgrm
typeDecl \Derive
  \identifier{type} \trm{is} typeExpr \opt{\trm{size} expression}
\egrm

\bgrm
typeExpr \Derive
     enumerationType  
\alt subrangeType     
\alt recordType       
\alt arrayType        
\alt \identifier{type}
\egrm

\bgrm  
enumerationType \Derive
  \verb'{' enumElement \zom{\trm{,} enumElement} \verb'}'
\egrm

\bgrm
enumElement \Derive
     \identifier{constant}        
\alt \numb           
\egrm

\bgrm
subrangeType \Derive
  expression \trm{..} expression
\egrm

\bgrm   
recordType \Derive
  \verb'{|' variableDecl \zom{\trm{;} variableDecl} \verb'|}'
\egrm

\bgrm
arrayType \Derive
  \trm{array} typeExpr \trm{of} typeExpr
\egrm

\subsection{Constant Declarations}
\bgrm                       
constantDeclSection \Derive
  \trm{const} constantDecl \zom{\trm{;} constantDecl}
\egrm

\bgrm
constantDecl \Derive
  \identifier{constant} \trm{:} \identifier{type} \opt{\trm{is} expression}
\egrm

\subsection{Function and Procedure Declarations}
\bgrm
functionDeclSection \Derive
  \trm{function} functionDecl \zom{\trm{;} functionDecl}
\egrm

\bgrm
functionDecl \Derive
  \identifier{function} \trm{(} \opt{formalParameter \zom{\trm{;} 
  formalParameter}} \trm{)} 
  \trm{:} \identifier{type} 
  \\ \qquad \trm{is} \opt{bounds} \opt{variableDeclSection} statementPart
\egrm

\bgrm
procedureDeclSection \Derive
  \trm{procedure} procedureDecl \zom{\trm{;} procedureDecl}
\egrm

\bgrm
procedureDecl \Derive
  \identifier{procedure} \trm{(} \opt{formalParameter \zom{\trm{;} 
  formalParameter}} \trm{)} 
  \\ \qquad \trm{is} \opt{bounds} \opt{\opt{variableDeclSection} statementPart}
\egrm

\bgrm
formalParameter \Derive
  \opt{parameterMode} \identifier{variable} \trm{:} \identifier{type}
\egrm

\bgrm
parameterMode \Derive
     \trm{in}      
\alt \trm{out}     
\alt \trm{inout}   
\egrm

\bgrm
bounds \Derive
  \trm{bounds} bound \trm{;} bound
\egrm

\bgrm
bound \Derive
     expression
\alt \trm{\~{}}         
\egrm

\subsection{Variable Declarations}
\bgrm
variableDeclSection \Derive
  \trm{var} variableDecl \zom{\trm{;} variableDecl}
\egrm

\bgrm
variableDecl \Derive
  \identifier{variable} \trm{:} typeExpr 
\egrm

\section{Expressions}
\bgrm
assignableExpression \Derive
     expression
\alt \trm{any} \identifier{type}
\egrm

\bgrm
expression \Derive
     designator        
\alt constantLiteral   
\alt functionCall                
\alt \trm{-} expression 
\alt expression \trm{*} expression
\alt expression \trm{/} expression
\alt expression \trm{+} expression
\alt expression \trm{-} expression 
\alt expression \trm{mod} expression
\alt expression \trm{=} expression
\alt expression \trm{/=} expression
\alt expression \trm{<} expression
\alt expression \trm{<=} expression
\alt expression \trm{>} expression
\alt expression \trm{>=} expression
\alt \trm{not} expression
\alt expression \trm{and} expression
\alt expression \trm{or} expression
\alt \trm{(} expression \trm{)}
\egrm

\bgrm
designator \Derive
     \identifier{variable}   
\alt designator \trm{.} \identifier{field} 
\alt designator \trm{[} expression \trm{]}
\egrm

\bgrm
constantLiteral \Derive
     \trm{\unitval}            
\alt \trm{false}
\alt \trm{true}
\alt \numb         
\egrm

\bgrm
fieldAssignment \Derive
  identifier \trm{=} expression
\egrm

\bgrm
functionCall \Derive
  \identifier{function} \trm{(} \opt{expression \zom{\trm{,} expression}} \trm{)}
\egrm

\section{Statements}
\bgrm
statementPart \Derive
  \trm{begin} statement \trm{end}  
\egrm

\bgrm                               
statement \Derive
  atomicStatement \trm{;} statement              
\alt atomicStatement         
\egrm

\bgrm
atomicStatement \Derive
     \trm{skip}               
\alt assignmentStatement      
\alt procedureCall         
\alt returnStatement       
\alt ifStatement           
\alt doStatement           
\egrm

\subsection{Assignment statement and Procedure Call}
\bgrm
assignmentStatement \Derive
  designator \trm{:=} assignableExpression
\egrm

\bgrm
procedureCall \Derive
  \identifier{procedure} \trm{(} \opt{expression \zom{\trm{,} expression}} \trm{)}
\egrm

\subsection{Return statement}
\bgrm
returnStatement \Derive
  \trm{return} expression   
\egrm

\subsection{If statement and Repetition}
\bgrm
ifStatement \Derive
  \trm{if} \oom{guardedCommand} \trm{fi}
\egrm

\bgrm
doStatement \Derive
  \trm{do} \oom{guardedCommand} \trm{od} 
\egrm

\bgrm
guardedCommand \Derive
  \trm{::} expression \trm{=>} statement
\egrm

















