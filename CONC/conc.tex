% $Id: concandfurther.tex,v 1.1 1997/10/19 19:37:17 davek Exp $
\chapter{Conclusions and Further Work}\label{chap:conc}
\section{Conclusions}
Formal methods can be useful for gaining confidence in the correct
behaviour of systems. Expressive languages and automatic analysis
techniques are needed to promote the acceptance of formal methods in
industry. For embedded systems, such languages and techniques should
allow the expression of, and reasoning about, real-time
properties. For a large class of embedded systems, broadcast
communication is an implementation primitive and should be accommodated
comfortably within a formal method intended for application in that
domain. This dissertation has proposed a formal language which is
claimed to satisfy these requirements, at least partially. Our
approach has been to define a language in which process behaviour can
be described using a few primitive operators, including operators for
the sending and receiving of broadcast messages. The communication
semantics is an abstraction of the CAN protocol and models both
message priority and communication latency. This language has proved
suitable as a bridge between the high-level expression of embedded
system models and their low-level representation in a form which is
appropriate for the application of a wide variety of analysis
techniques. We have demonstrated how the most efficient of current
methods can be applied to models expressed in our language. In
particular, we have given an algorithm for on-the-fly generation of
the simulation graph, including clock activity reduction. This
provides a foundation for the application of methods such as
reachability analysis, model checking, TBA emptiness, minimisation and
time-abstract bisimulation, as implemented in tools such as KRONOS,
UPPAAL and CADP. In addition, we have demonstrated the use of
minimised automata for compact state space representation. Minimised
automata have been employed in the model checking tool SPIN for the
analysis of untimed, asynchronous models.  They are applied here, for
the first time, in the analysis of timed system models and we give
experimental data to confirm their utility.
   
 
\section{Further Work}
The expressiveness of our language is restricted in several ways. For
example, we do not allow control to
depend explicitly on the time of event occurrences, nor is it possible for
an interrupted task to resume
execution from its point of interruption. Both features are available
in ET-LOTOS~\cite{her:98}, for example. More seriously, we cannot
model multi-tasking systems in which the CPU resource is allocated to
tasks using a more sophisticated scheduling policy than round-robin,
e.g., a fixed priority preemptive policy. It also remains to consider
the modelling of the occurrence of faults in broadcast
message transmission.

It is not difficult to see how some of these additional features could
be accommodated, e.g. an explicit scheduler could be added to the
execution model, as could a `daemon' for fault injection. The problem
is to cope with the extra complexity and its effect on state
explosion. The addition of such features almost certainly leads to a
hybrid system model for which many verification problems become either
undecidable or, at best, even more resource demanding~\cite{hkpv:95}.
 
Even without adding to the complexity of the language, further work is
needed on state space explosion. Some obvious lines of inquiry are
suggested at the end of Chapter~\ref{chap:sggen}, where work on
variable ordering and live variable analysis has the potential to
bring reductions in the size of the discrete state space. Also of
interest is investigation of the use of partial order reduction
and symbolic clock constraint representations.  In particular, research
is needed to compare the performance of DDD's with that of
MA's when applied to typical asynchronous broadcast systems,
especially when considered in combination with reduction techniques
such as partial order and inclusion/convex hull abstractions, where
the use of MA's seems to offer a \emph{prima facie} advantage.

One can imagine that more use can be made of the high-level, algebraic
structure apparent in the models, to transform them into more
space-efficient, equivalent representations. This should be possible
at all levels, from the high-level \candle\ model, through the
\bcandle\ and net representations, to the timed automaton.
Undoubtedly, compositional techniques will be required in order to extend a
fully automated approach to industrial-scale systems. 

Finally, further work remains on a number of pragmatic issues
affecting industrial usage of the technology: at a high-level, the
issue of requirements capture and their relationship to formal
specifications; at a low-level, the formal specification and
implementation of an execution environment which satisfies the
abstraction assumptions of Chapter~\ref{chap:bcandle}.  

Some progress has been made but much remains to be done before it will
be possible to realise Pnueli's vision of a \emph{seamless development
process}~\cite{pnu:99} for broadcasting embedded control systems.


 
